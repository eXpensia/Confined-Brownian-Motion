\section*{Abstract}
In this manuscript, I present the work done during my PhD on confined Brownian motion. Brownian motion is the erratic movement of microscopic particles when immersed into a fluid. Thanks to Einstein and followers, it is generally possible to describe Brownian motion using simple equations. However, in the last two decades a scientific revolution has taken place with the advent of miniaturization and in particular microfluidics, thanks to which we are now able to create complex networks of pipes at the micrometer scale. Microfluidics makes it possible to sort particles, such as drops, cells or bubbles, but also to distribute drugs in cells and observe their effect on thousands of them. Regarding Brownian motion, it has been observed that once confined, a particle moves much slower: the laws discovered by Einstein must be modified to take into account the confinement-induced effects.
My thesis work consists in experimentally measuring, analyzing and modeling the movement of micrometric colloids that move randomly near a wall. To do this I use Mie holography: I illuminate particles with a laser and thanks to the interaction between the laser and the small portion of light scattered by the colloid I obtain interference patterns. This method allows reconstructing the trajectory of a particle with a resolution of the nanometer. Then, I build a statistical inference tool that permits to extract from trajectories important informations such as equilibrium potentials or surface forces.  Our goal for the future is to use this novel tool to allow the in-depth study of the motility of microbiological entities, such as stem cells for example, whose differentiation mechanisms are still not understood, or cancerous tumors and their abnormal migration properties.

\paragraph*{Structure of the manuscript}\mbox{}\\
In chapters 1 and 2, I present a general introduction of the subject both in English and French. In chapter \ref{sec:chapter1}, I present the history of Brownian motion, starting from how it has been discovered to its numerical simulation. In chapter \ref{sec:chapter2}, I present state-of-the-art methods that permit tracking single colloids. In particular, I detail how we use the Lorenz-Mie framework to optically record the thermally-induced tridimensional trajectories of individual microparticles, within salty aqueous solutions, in the vicinity of a rigid wall, and in the presence of surface charges. Then, I show the viability of this framework by studying a well-known problem: the sedimentation of a colloid. In chapter \ref{chap3}, I construct the time-dependent position and displacement probability density functions, and study the non-Gaussian character of the latter which is a direct signature of the hindered mobility near the wall. Based on these distributions, I implement a novel, robust and self-calibrated multifitting method, allowing for the thermal-noise-limited inference of diffusion coefficients spatially-resolved at the nanoscale, equilibrium potentials, and forces at the femtonewton resolution. Finally, in chapter \ref{chapfin}, I use this novel tool in order to infer non-conservative forces and study long-time higher-order statistical properties.

\newpage

\section*{Résumé}

Dans ce manuscrit, je présente le travail effectué pendant mon doctorat sur le mouvement brownien en confinement et aux interfaces, qui est une situation canonique, rencontrée de la biophysique fondamentale à l'ingénierie à l'échelle nanométrique.  Notre objectif est d'utiliser les fluctuations thermiques afin de mesurer des quantités physiques importantes telles que les forces de surface et la mobilité avec une grande précision.


Dans une première partie, je présente l'histoire du mouvement brownien.  Depuis sa découverte jusqu'à sa simulation numérique. Dans une deuxième partie, je présente le microscope sur mesure construit pendant mon doctorat.  Puis, je détaille l'utilisation du framework Lorenz-Mie pour l'enregistrement optique des trajectoires tridimensionnelles thermiquement induites de microparticules individuelles, dans des solutions aqueuses salées, à proximité d'une paroi rigide, et en présence de charges de surface. Ensuite, je construis les fonctions de densité de probabilité de position et de déplacement en fonction du temps. J'étudie en suivant, le caractère non gaussien de ces dernières, qui est une signature directe de la modification de la mobilité près de la paroi.  Sur la base de ces distributions, nous mettons en œuvre une nouvelle méthode, robuste et auto-calibrée de multifitting, permettant l'inférence limitée par le bruit thermique des coefficients de diffusion résolus spatialement à l'échelle nanométrique, des potentiels d'équilibre et des forces à la résolution femtonewton. Enfin, dans la dernière partie, j'utilise ce nouvel outil pour déduire les forces non-conservatives et étudier les propriétés statistiques d'ordre supérieur à long terme.




