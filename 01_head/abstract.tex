\section*{Abstract}

In this manuscript, I present the work done during my phD on Brownian motion in confinement and at interfaces, which is a canonical situation, encountered from fundamental buiphysics to nanoscale engineering. Our goal is to use the thermal fluctuation in order to measure important physical quantities such as surface forces and mobility with an accuracy.

In a first part I present the history of the Brownian motion. Starting from how it has been discovered to its numerical simulation. In a second part, I present the custom-built microscope built during my phD. Then I detail how we use the Lorenz-Mie framework to optically record the thermally-induced tridimensional trajectories of individual microparticles, within salty aqueous solutions, in the vicinity of a rigid wall, and in the presence of surface charges. Then, I construct the time-dependent position and displacement probability density functions, and study the non-Gaussian character of the latter which is a direct signature of the hindered mobility near the wall. Based on these distributions, we implement a novel, robust and self-calibrated multifitting method, allowing for the thermal-noise-limited inference of diffusion coefficients spatially-resolved at the nanoscale, equilibrium potentials, and forces at the femtonewton resolution. Finally, in the last part, I use this novel tool in order to infer non-conservative forces and study long-time higher-order statistical properties.


\section*{Résumé}

Dans ce manuscrit, je présente le travail effectué pendant mon doctorat sur le mouvement brownien en confinement et aux interfaces, qui est une situation canonique, rencontrée de la biophysique fondamentale à l'ingénierie à l'échelle nanométrique.  Notre objectif est d'utiliser les fluctuations thermique afin de mesurer des quantités physiques importantes telles que les forces de surface et la mobilité avec une grande précision.


Dans une première partie, je présente l'histoire du mouvement brownien.  Depuis sa découverte jusqu'à sa simulation numérique.Dans une deuxième partie, je présente le microscope sur mesure construit pendant mon doctorat.  Puis je détaille comment nous utilisons le cadre de Lorenz-Mie pour enregistrer optiquement les trajectoires tridimensionnelles thermiquement induites de microparticules individuelles, dans des solutions aqueuses salées, à proximité d'une paroi rigide, et en présence de charges de surface.Ensuite, je construis les fonctions de densité de probabilité de position et de déplacement en fonction du temps, et j'étudie le caractère non gaussien de ces dernières, qui est une signature directe de la mobilité entravée près de la paroi.  Sur la base de ces distributions, nous mettons en œuvre une méthode nouvelle, robuste et auto-calibrée de multifitting, permettant l'inférence limitée par le bruit thermique des coefficients de diffusion résolus spatialement à l'échelle nanométrique, des potentiels d'équilibre et des forces à la résolution femtonewton. Enfin, dans la dernière partie, j'utilise ce nouvel outil pour déduire les forces non-conservatives et étudier les propriétés statistiques d'ordre supérieur à long terme.


