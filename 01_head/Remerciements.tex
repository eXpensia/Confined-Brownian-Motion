\section*{Abstract}
In this manuscript, I present the work done during my PhD on confined Brownian motion. Brownian motion is the erratic movement of microscopic particles when immersed into a fluid. Thanks to Einstein and his successors, it is generally possible to describe Brownian motion using simple equations. However, in the last two decades a scientific revolution has taken place with the advent of miniaturization and in particular microfluidics, enabling the creation of complex networks of pipes at the micrometer scale. Microfluidics makes it possible to sort out particles, such as drops, cells or bubbles, but also to distribute drugs in cells and observe their effect on thousands of them. Regarding Brownian motion, it has been observed that once confined near a wall, a particle moves much slower due to non-slip boundary conditions at the wall. The mobility is thus modified by confinement-induced effects.

My thesis work consists in experimentally measuring, analyzing and modeling the movement of micrometric colloids diffusing near a wall. To track the motion of confined Brownian microparticles, I use Lorenz-Mie holography. The Lorenz-Mie framework allows me to record the thermally-induced three-dimensional trajectories of individual microparticles, within salty aqueous solutions, in the vicinity of a rigid wall, and in the presence of surface charge with a nanometric resolution. From the recorded trajectory, I construct the time-dependent position and displacement probability density functions, and analyze the non-Gaussian character of the latter which is a direct signature of the hindered mobility near the wall. Based on these distributions, I implement a novel, robust and self-calibrated multifitting method, allowing thermal-noise-limited inference of diffusion coefficients spatially-resolved at the nanoscale, equilibrium potentials, and forces at the femtonewton resolution. Moreover, I use this novel tool to deduce non-conservative forces and study long-time higher-order statistical properties. Our objective for the future is to use this novel tool to have a new approach in various problems relevant to nanophysics and microbiology.


\newpage

\section*{Résumé}

Dans ce manuscrit, je présente le travail effectué pendant mon doctorat sur le mouvement brownien confiné. Le mouvement brownien est le mouvement erratique de particules microscopiques lorsqu'elles sont immergées dans un fluide. Grâce à Einstein et ses successeurs, il est généralement possible de décrire le mouvement brownien à l'aide d'équations simples. Cependant, au cours des deux dernières décennies, une révolution scientifique a eu lieu avec l'avènement de la miniaturisation et en particulier de la microfluidique, permettant la création de réseaux complexes de tuyaux à l'échelle micrométrique. La microfluidique permet de trier des particules, comme des gouttes, des cellules ou des bulles, mais aussi de distribuer des médicaments dans des cellules et d'observer leur effet sur des milliers d'entre elles. En ce qui concerne le mouvement brownien, il a été observé qu'une fois confinée près d'une paroi, une particule se déplace beaucoup plus lentement en raison des conditions de non-glissement à la paroi. La mobilité est donc modifiée par les effets induits par le confinement.

Mon travail de thèse consiste à mesurer, analyser et modéliser expérimentalement le mouvement de colloïdes micrométriques diffusant près d'une paroi. Pour suivre le mouvement de microparticules browniennes confinées, j'utilise l'holographie de Lorenz-Mie. Le cadre de Lorenz-Mie me permet d'enregistrer les trajectoires tridimensionnelles thermiquement induites de microparticules individuelles, dans des solutions aqueuses salées, à proximité d'une paroi rigide, et en présence d'une charge de surface avec une résolution nanométrique. A partir de la trajectoire enregistrée, je construis les fonctions de densité de probabilité de position et de déplacement en fonction du temps, et j'analyse le caractère non-gaussien de ces dernières qui est une signature directe de la mobilité modifiée près de la paroi. Sur la base de ces distributions, je mets en œuvre une nouvelle méthode d'ajustement multiple, robuste et auto-calibrée, permettant l'inférence limitée par le bruit thermique des coefficients de diffusion résolus spatialement à l'échelle nanométrique, des potentiels d'équilibre et des forces résolus au femtonewton. De plus, j'utilise ce nouvel outil pour déduire les forces non-conservatives et étudier les propriétés statistiques d'ordre supérieur à long terme. Notre objectif pour l'avenir est d'utiliser ce nouvel outil pour avoir une nouvelle approche dans divers problèmes liés à la nanophysique et à la microbiologie.



\newpage

\begin{figure}[h]
	\begin{center}
		\includegraphics{01_head/images/RENDU_16_10_MAX.png}
		\caption{Artist's view of the Lorenz-Mie method. A polystyrene ball (in white) is illuminated from above. The interference pattern is indicated by the concentric rings (Credit: \href{https://www.behance.net/pierresavary357}{Pierre Savary}). \\ Vue d'artiste de la méthode de Lorenz-Mie. Une bille de polystyrène (en blanc) est éclairée par le haut. La figure d'interférences est indiquée par les anneaux concentriques (Crédit : \href{https://www.behance.net/pierresavary357}{Pierre Savary}).}
	\end{center}
\end{figure}

