
\newpage
\section*{Remerciements}
Avant de commencer, je souhaiterais remercier Christophe Ybert, Aloïs Wurger, Lionel Bureau et Frédéric Restagno d’avoir accepté de faire partie de mon jury, et pour leurs commentaires sur ce manuscrit.

Ce travail au cours des trois dernières années n’aurait pas été possible sans toutes les personnes avec qui j’ai travaillé. 

Je souhaiterais remercier mon directeur de thèse Thomas Salez de m’avoir accueilli dans son groupe et partager ses connaissances, son optimisme et son envie de découvrir de nouvelles choses. Merci aussi pour tous tes  « soft skills » : ta capacité d’écoute et ta façon de diriger l’équipe.

De la même manière, je souhaiterais remercier mon co-encadrant Yacine Amarouchene, d’avoir partagé avec moi toute sa connaissance de la physique expérimentale. Je me souviendrai toujours des longues discussions que nous avons eues devant les données afin de les déchiffrer. 

Ensuite, je souhaiterais remercier les différents membres de l’équipe, Louis Bellando De Castro, Julien Brugin, Yann Louyer, Nicolas Farès, Caroline Kopecz-Muller , Vincent Bertin et Élodie Millan ainsi que Yonah Grondin que j’ai eu la chance d’encadrer en stage. Pour toutes discussions pour faire avancer la recherche qui est un grand travail collectif. Ce travail n’aurait pas non plus été possible sans les équipes techniques du LOMA, plus particulièrement merci Étienne Harté de nous avoir prêté l’objectif et Josiane Parzych pour l’organisation administrative qui me dépasse complètement.

Je souhaiterais aussi remercier les membres du laboratoire Gulliver à Paris et plus particulièrement Olivier Dauchot de m’avoir permis de venir travailler 2 mois et de découvrir une nouvelle façon de travailler. Et bien sûr, Joshua McGraw, Alexandre Viquin et Gabriel Guyard de m’avoir accueilli et avec qui j’ai passé un super moment. Une mention spéciale pour la journée road trip à Denver !



Je tiens à remercier NAG et BanBan pour toutes les discussions et ça depuis la licence ! Tous mes amis et ma famille pour leurs soutiens.

Enfin j’aimerais remercier Élodie de partager la vie avec moi et de m’avoir soutenu au cours de ces trois dernières années et en particulier pendant l’écriture de ce manuscrit. Ses souvenirs seront gravés à tout jamais dans ma mémoire.


