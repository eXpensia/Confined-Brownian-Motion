\section{Introduction}
  \label{sec:Intro}

Since the observations of Gordon Moore in the 60's we know that the technological progress is bound to our ability to miniaturize. It's indeed due to the miniaturization that we are able to have more computational power leading to the rise of knew technologies like the Deep Learning that showed the need of large computational capabilities by having the computer program \textit{AlphaGo} beating \textit{Lee Sedol} one of the greatest player of \textit{Go} in 2016. Since this powerful demonstration AIs using the same technologies are showing up in every field, from the language translator to autonomous cars and is know starting to be extensively used in physics with in 2020 the first focus session on machine learning at the \textit{March Meeting} that continued this year with presentations at every sessions.
The success of Deep learning is not due to the fact that it's new and fancy algorithm since it known for several decade but only the fact that the miniaturization permitted to do the stunning amount of computation needed to have a smart AI. Our ability to use this technologies is finally bound to our ability to understand the surface physics at the manometer scale.

On another side we have microfluidic since the 80s which is an incredible multidisciplinary field involving chemistry, engineering, soft matter physics and also biotechnology. Microfluidic permitted the development of daily life technologies like the ink-jet printers or more advanced tools such as DNA chips or lab-on-a-chip technology. The ability to compose with a lot of different system to build microfluidic systems is a wonderful playground for physicists which gave a lot of complex systems in confinement to study and understand how different boundaries can change the dynamic properties of a system. At a time of miniaturization and nanotechnologies, the need of tools permitting the systematic study of complex confined system is a key. 

In order to address these challenges my work in the past three years focused on using the confined Brownian motion. Brownian Motion is a central paradigm in modern science. It has implications in fundamental physics, biology, and even finance, to name a few. By understanding that the apparent erratic motion of colloids is a direct consequence of the thermal motion of surrounding fluid molecules, pioneers like Einstein and Perrin provided decisive evidence for the existence of atoms, Specifically, free Brownian motion in the bulk us characterized by a typical spatial extent evolving as the square root of time, as well as Gaussian displacements. At a time of miniaturization and interfacial science, and moving beyond the idealized bulk picture, it is relevant to consider the added roles of boundaries to the above context. Indeed, Brownian motion at interfaces and in confinement is a widespread practical situation in microbiology and nanofluidics. In such case, surface effects become dominant and alter drastically the Brownian statistics, with key implications towards: i) the understanding and smart control of the interfacial dynamics of microscale entities; and ii) high-resolution measurements of surfce forces at equilibrium.
Intertingly, a confined colloid will exhibit non-Gaussian statistics in displacements, due to the presence of multiplicative noises induced by the hindered mobility near the wall. Besides, the particle can be subjected to electrostatic or Van der Waals forces exerted by the interface, and might experience slippage too. Considering the two-body problem, the nearby boundary can also induce some effective interaction. Previous studies have designed novel methods to measure the diffusion coefficient of confined colloids, or to infer surface forces.

In the the first part of the manuscript 