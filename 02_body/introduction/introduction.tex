\section{Introduction}
  \label{sec:Intro}




\subsection{Open access}

A great part of my work depends on resources that are in open-access (OA). OA resources range from articles that can provide experimental data or the source code used for simulation, from entire framework to analyze data or build state-of-the-art simulations. I myself, wants to promote open access research as it enables the fastest spread of the latest techniques around the globe by also removing the costs barriers. Thus, all my manuscript and source codes for the plotting figures are uploaded on a free accessible Github repository. Along this manuscript, Github logos (\href{https://github.com/eXpensia/Ma-these/}{\faGithub}) can be found, these are hyperlinks to aforementioned information. The links can be found especially in the figures captions and lead to its source code, or along the text to redirect to any OA resources I am referring to. All the figures, are done using Python and Jupyter notebooks, thus providing the easiest way to reproduce or reuse them. Therefore, the interested reader can follow along the manuscript and change the different variables to see how physics changes.  An explanation on how to setup-up a Python environment and employ the notebooks using Conda can be found on my repository \href{https://github.com/eXpensia/Ma-these/}{\faGithub}.


\vspace{4cm}

\begin{figure*}[h]
	\begin{center}
		\includegraphics[width=16cm]{02_body/introduction/image/perrin.jpg}
	\end{center}
\end{figure*}