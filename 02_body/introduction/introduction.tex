\section{Introduction}
  \label{sec:Intro}



Brownian motion is a central paradigm in modern science. It has implications in fundamental physics, biology, and even finance, to name a few. By understanding that the apparent erratic motion of colloids is a direct consequence of the thermal motion of surrounding fluid molecules, pioneers like Einstein and Perrin provided decisive evidence for the existence of atoms~\cite{einstein_uber_1905,perrin_les_2014}.

During the past 30 years~\cite{whitesides_origins_2006, convery_30_2019} microfluidics and the development of \textit{lab-on-a-chip} technologies \cite{neuzil_revisiting_2012,ding_surface_2013} became standard in chemistry, biology and medicine. Thus, at a time of miniaturization and interfacial science, and moving beyond the idealized bulk picture, it is relevant to  consider the added roles of boundaries to the above context. Indeed, Brownian motion at interfaces and in confinement is a widespread practical situation in microbiology and nanofluidics. In such a case, surface effects become dominant and alter drastically the Brownian statistics, with key implications towards: i) the understanding and smart control of the interfacial dynamics of microscale entities; and ii) high-resolution measurements of surface forces at equilibrium. Interestingly, a confined colloid will exhibit non-Gaussian statistics in displacements, due to the presence of multiplicative noises induced by the hindered mobility near the wall~\cite{felderhof_effect_2005,wang_anomalous_2009,chechkin_brownian_2017}. Besides, the particle can be subjected to electrostatic or Van der Waals forces~\cite{bouzigues_nanofluidics_2008} exerted by the interface, and might experience slippage too~\cite{joly_probing_2006,mo_brownian_2017}. Previous studies have designed novel methods to measure the diffusion coefficient of confined colloids~\cite{faucheux_confined_1994,dufresne_brownian_2001,carbajal-tinoco_asymmetry_2007,eral_anisotropic_2010,sharma_high-precision_2010,mo_broadband_2015,matse_test_2017}, or to infer surface forces~\cite{prieve_measurement_1999,banerjee_experimental_2005,sainis_statistics_2007,volpe_influence_2010,wang_measurement_2011,li_subfemtonewton_2019}. However, such a statistical inference is still an experimental challenge, and a precise calibration-free method taking simultaneously into account the whole ensemble of relevant properties, over broad spatial and time ranges, is currently lacking.

In my thesis, I aimed at filling the gap identified above by implementing a novel method of statistical inference on a set of trajectories of individual microparticles recorded by holographic microscopy. In the chapter 3, I introduce Brownian motion from its discovery to its mathematical description and numerical simulation. In the chapter 4, I present our state-of-the-art particle-tracking method. In the chapter 5, I further use it experimentally to study  buoyant particles that are free to evolve within salty aqueous solutions, near a rigid substrate, and in the presence of surface charges. Besides, I present an optimization scheme to determine precisely all the physical parameters and the actual distance to the wall, at once. All together, this procedure leads to the robust calibration-free inference of the two central quantities of the problem: i) the space-dependent short-term diffusion coefficients, with a nanoscale spatial resolution; and ii) the total force experienced by the particle, at the thermal-noise limited femtonewton resolution. Finally, in the chapter 6, I use this novel tool in order to infer non-conservative forces and study long-time higher-order statistical properties.



\subsection{Open access}

A great part of my work depends on resources that are in open-access (OA). OA resources range from articles that can provide experimental data or the source code used for simulation, to entire frameworks to analyze data or build state-of-the-art simulations. I, want to promote open access research as it enables the fastest spread of the latest techniques around the globe by removing the cost barriers. Thus, my manuscript and source codes are uploaded on a free accessible Github repository. Along this manuscript, Github logos (\href{https://github.com/eXpensia/Ma-these/}{\faGithub}) can be found. These are hyperlinks to aforementioned information. The links can be found especially in the figure captions and lead to their source code, or along the text to redirect to any OA resource I am referring to. All the figures, are done using Python and Jupyter notebooks, thus providing an easy way to reproduce or reuse them. The interested reader can follow along the manuscript and change the different variables to see how physics changes.  An explanation on how to setup a Python environment and employ the notebooks using Conda can be found in my repository \href{https://github.com/eXpensia/Ma-these/}{\faGithub}.



\begin{figure}[h]
	\begin{center}
		\includegraphics[width=16cm]{02_body/introduction/image/perrin.jpg}
		\caption{French physicist and Nobel laureate Jean Perrin (1870-1942) with his ``mega-spectroscope" at the Institut Curie in 1927.}
	\end{center}
\end{figure}


\section{Introduction française}



Le mouvement brownien est un paradigme central de la science moderne. Il a des implications en physique fondamentale, en biologie et même en finance, pour n'en citer que quelques-unes. En comprenant que le mouvement erratique apparent des colloïdes est une conséquence directe du mouvement thermique des molécules du fluide environnant, des pionniers comme Einstein et Perrin ont fourni des preuves décisives de l'existence des atomes~\cite{einstein_uber_1905,perrin_les_2014}.

Au cours des 30 dernières années~\cite{whitesides_origins_2006, convery_30_2019} la microfluidique et le développement des technologies \textit{lab-on-a-chip} \cite{neuzil_revisiting_2012,ding_surface_2013} sont devenus des standards en chimie, biologie et médecine. Ainsi, à l'heure de la miniaturisation et de la science interfaciale, et au-delà de l'image idéalisée du volume, il est pertinent de considérer le rôle supplémentaire des bords dans le contexte ci-dessus. En effet, le mouvement Brownien aux interfaces et en confinement est une situation pratique répandue en microbiologie et en nanofluidique. Dans un tel cas, les effets de surface deviennent dominants et modifient radicalement les statistiques browniennes, avec des implications clés pour : i) la compréhension et le contrôle intelligent de la dynamique interfaciale des entités à l'échelle microscopique ; et ii) les mesures à haute résolution des forces de surface à l'équilibre. Il est intéressant de noter qu'un colloïde confiné présentera des statistiques non gaussiennes dans les déplacements, en raison de la présence de bruits multiplicatifs induits par une mobilité modifiée près de la paroi~\cite{felderhof_effect_2005,wang_anomalous_2009,chechkin_brownian_2017}. En outre, la particule peut être soumise à des forces électrostatiques ou de Van der Waals~\cite{bouzigues_nanofluidics_2008} exercées par l'interface, et peut également subir un glissement~\cite{joly_probing_2006,mo_brownian_2017}. Des études antérieures ont conçu de nouvelles méthodes pour mesurer le coefficient de diffusion des colloïdes confinés~\cite{faucheux_confined_1994,dufresne_brownian_2001,carbajal-tinoco_asymmetry_2007,eral_anisotropic_2010,sharma_high-precision_2010, mo_broadband_2015,matse_test_2017}, ou pour déduire les forces de surface~\cite{prieve_measurement_1999,banerjee_experimental_2005,sainis_statistics_2007,volpe_influence_2010,wang_measurement_2011,li_subfemtonewton_2019}. Cependant, une telle inférence statistique reste un défi expérimental, et une méthode précise sans calibration prenant en compte simultanément l'ensemble des propriétés pertinentes, sur de larges plages spatiales et temporelles, fait actuellement défaut.

Dans ma thèse, j'ai cherché à combler le vide identifié ci-dessus en mettant en œuvre une nouvelle méthode d'inférence statistique sur un ensemble de trajectoires de microparticules individuelles enregistrées par microscopie holographique. Dans le chapitre 3, je présente le mouvement brownien depuis sa découverte jusqu'à sa description mathématique et sa simulation numérique. Dans le chapitre 4, je présente notre méthode de pointe pour le suivi des particules. Dans le chapitre 5, je l'utilise expérimentalement pour étudier les particules flottantes qui sont libres d'évoluer dans des solutions aqueuses salées, près d'un substrat rigide, et en présence de charges de surface. En outre, je présente un schéma d'optimisation pour déterminer précisément tous les paramètres physiques et la distance réelle à la paroi, en une seule fois. Dans l'ensemble, cette procédure conduit à l'inférence robuste et sans calibration des deux quantités centrales du problème : i) les coefficients de diffusion à court terme dépendant de l'espace, avec une résolution spatiale à l'échelle nanométrique ; et ii) la force totale subie par la particule, résolu au femtonewton limitée par le bruit thermique. Enfin, dans le chapitre 6, j'utilise ce nouvel outil pour déduire les forces non-conservatives et étudier les propriétés statistiques d'ordre supérieur à long terme.




\subsection{Libre Accès}

Une grande partie de mon travail dépend de ressources en libre accès (OA). Les ressources en libre accès vont des articles qui peuvent fournir des données expérimentales ou le code source utilisé pour des simulations, à des \textit{framework} entiers pour analyser les données ou construire des simulations de pointe. Je souhaite promouvoir la recherche en libre accès, car elle permet une diffusion plus rapide des dernières techniques dans le monde entier en supprimant les obstacles financiers. Ainsi, mon manuscrit et mes codes sources sont téléversée sur un dépôt Github en libre accès. Le long de ce manuscrit, on peut trouver des logos Github (\href{https://github.com/eXpensia/Ma-these/}{\faGithub}). Il s'agit d'hyperliens vers les informations susmentionnées. Les liens se trouvent notamment dans les légendes des figures et mènent à leur code source, ou le long du texte pour rediriger vers toute ressource OA à laquelle je fais référence. Toutes les figures sont réalisées à l'aide de Python et de Jupyter notebooks, ce qui permet de les reproduire ou de les réutiliser facilement. Le lecteur intéressé peut suivre le manuscrit et changer les différentes variables pour voir comment la physique change.  Une explication sur la façon de configurer un environnement Python et d'utiliser les \textit{notebooks} en utilisant Conda peut être trouvée dans mon dépôt de données \href{https://github.com/eXpensia/Ma-these/}{\faGithub}.



\begin{figure}[h]
	\begin{center}
		\includegraphics[width=16cm]{02_body/introduction/image/perrin.jpg}
		\caption{Le physicien français et lauréat du prix Nobel Jean Perrin (1870-1942) avec son ``méga-spectroscope" à l'Institut Curie, en 1927}
	\end{center}
\end{figure}