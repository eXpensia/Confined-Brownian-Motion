











































\section{Conclusion}

In this manuscript, we have covered a broad spectrum of the Brownian motion. We started by its history by presenting its first observation in the 19th by Robert Brown and its mathematical description, and numerical simulation. We have then discribed different techniques that permits tracking of individual microparticles. We mainly focused on the Lorenz-Mie framework since as we used it to study the free confined Brownian motion. We indeed focused on the Lorenz-Mie framzork as it requieres no calibration by measuring the radius and optical index of each tracked particles. We then showed the experimental setup, which is a custom made inverted microscope that we optized over the years to capture Lorenz-Mie holograms. To retrieve the trajectory from the captured holograms we mainly used the tool developped at the Grier's Lab under the name of Pylorenzmie \href{https://github.com/davidgrier/pylorenzmie}{\faGithub} and a custom version, Wraplorenzmie to automate the tracking across whole movies and simplify the use of MP4 files \href{https://github.com/eXpensia/wraplorenzmie}{\faGithub}. An example of the Jupyter notebooks employed for the tracking can be found in appendix \ref{app:tracking}.

The particles thus tracked and the trajcetories retrieve, we focused on the analyse of the confined Brownian motion. We first detailed how the physics changes starting from the DLVO interactions between the surface and the particles and hindered diffusion. Due to this hindered diffusion, a spurious drift appears while using the overdamped Langevin equation which needs to be taken into account while doing simulations, and later the force measurements.

Once the underlying theory has been explained, we described the experimental results and how the trajcetories are analized. Starting from static observables such as equilibrium distributions to dynamic observables with the Mean Squared Displacements and displacement distribution. In particular, the \gls{MSD} exhibit a normal distribution at short time, and a equilibrium regime for the movement perpendicular to the wall. As to short time displacement distribution, it shows non-Gaussian properties despite the normal distribution indicated by the \gls{MSD}.

The statistical properties of a confined particle understood and correctly measured, we then focused on measuring the hindered diffusion. This measure was done using a novel method devellopped by Frishman and Ronceray \cite{frishman_learning_2020} using information theory to infer the the diffusion onto a basis of function, leading to high accuracy near the surface where there is less data. 

All the experimental data, from the equilibrium distribution to the hindered diffusion can all be given by the parameters of the system $B$, $\ell _\mathrm{D}$ and $\ell_\mathrm{B}$. We thus built a multi-fitting method, that permits to infer precisely those parameters by taking into account all the observable at the same time. From this measurement, we measured precisely the equilibrium potential from which we can compute the conservative force which permitted to verify that we correctly measure the force measured though the local drifts.

Starting from the construction of the microscope to numerical analysis, built a multi-scale statistical analysis for the problem of freely diffusing individual colloids near a rigid wall. By combining all the observables, we are able to reach thermal-noise-limited femtoNewton spatial and force resolutions at the nanoscale. This work has finally been pusblished in Physical Review Research and is in appendix \ref{app:prr}.

As shown in the last chapter, we are currently using our multi-scale statistical analysis in order to inspect soft lubrication theory applicated to Brownian motion. Particularly, if the a Brownian particle can trigger elastohydrodynamic lift which is a non-conservative forces. Also we are working on pushing the limit of our resolution to measure fine effects on high order cumulents to experimentaly verify a theoritical model, that exhibit correlation between the perpendicular and parallel displcement of a free diffusing particle, which can be counterintuitive and has not yet been measured. Finally, we believe that the ability to measure tiny surface forces, locally, and at equilibrium, as well the possible extension of the method to non-conservative forces and out-of-equilibrium settings, opens fascinating perspectives for nanophysics and biophysics.

\section{Conclusion en français}

Dans ce manuscrit, nous avons couvert un large spectre du mouvement brownien. Nous avons commencé par son histoire en présentant sa première observation au 19ème siècle par Robert Brown, sa description mathématique ainsi que comment le simuler numériquement. Nous avons ensuite décrit les différentes techniques qui permettent de localiser des microparticules individuelles. Nous nous sommes principalement concentrés sur la théorie de Lorenz-Mie puisque nous l'avons utilisé pour étudier expérimentalement le mouvement brownien libre confiné. Nous nous sommes en effet concentrés la théorie de Lorenz-Mie car elle ne nécessite aucune calibration, en mesurant le rayon et l'indice optique de chaque particule localisée. Nous avons ensuite expliqué le dispositif expérimental, qui est un microscope inversé fait sur mesure, que nous avons optimisé au fil des ans pour capturer des hologrammes de Lorenz-Mie. Pour récupérer la trajectoire des hologrammes capturés, nous avons principalement utilisé l'outil développé au laboratoire de Grier sous le nom de Pylorenzmie \href{https://github.com/davidgrier/pylorenzmie}{\faGithub} et une version personnalisée, Wraplorenzmie pour automatiser la localisation sur des films entiers et simplifier l'utilisation des fichiers MP4 \href{https://github.com/eXpensia/wraplorenzmie}{\faGithub}. Un exemple des \textit{Jupyter Notebook} utilisés pour la localisation se trouve en annexe \ref{app:tracking}.

Les particules ainsi localisée et les trajcetories récupérées, nous nous sommes concentrés sur l'analyse du mouvement brownien confiné. Nous avons d'abord détaillé comment la physique change à partir des interactions DLVO entre la surface et les particules, et la diffusion locale. En raison de cette diffusion locale, un \textit{spurious drift} apparaît lors de la résolution de l'équation de Langevin suramortie qui doit être prise en compte lors des simulations, et plus tard lors des mesures de force.

Une fois la théorie sous-jacente expliquée, nous avons décrit les résultats expérimentaux et la manière dont les trajcetories sont analysées. En partant des observables statiques comme les distributions d'équilibre jusqu'aux observables dynamiques avec les déplacements quadratiques moyens et la distribution des déplacements. En particulier, les \gls{MSD} présentent une distribution normale aux temps courts, et un régime d'équilibre pour le mouvement perpendiculaire au mur. Quant à la distribution des déplacements à court terme, elle présente des propriétés non gaussiennes malgré la distribution normale indiquée par les \gls{MSD}.

Les propriétés statistiques d'une particule confinée étant comprises et correctement mesurées, nous nous sommes ensuite attachés à mesurer la diffusion locale. Cette mesure a été effectuée à l'aide d'une nouvelle méthode développée par Frishman et Ronceray \cite{frishman_learning_2020} utilisant la théorie de l'information pour inférer la diffusion sur une base de fonction, conduisant à une grande précision près de la surface où il y a moins de données. 

Toutes les données expérimentales, de la distribution à l'équilibre à la diffusion locale, peuvent toutes être données par les paramètres du système $B$, $\ell _\mathrm{D}$ et $\ell_\mathrm{B}$. Nous avons donc construit une méthode \textit{multi-fitting}, qui permet de déduire précisément ces paramètres en prenant en compte tous les observables en même temps. A partir de cette mesure, nous avons mesuré précisément le potentiel d'équilibre à partir duquel nous pouvons calculer la force conservative qui a permis de vérifier que nous mesurons correctement la force mesurée par les dérives locales.

De la construction du microscope à l'analyse numérique, nous avons construit une analyse statistique multi-échelle pour le problème de colloïdes individuels diffusant librement près d'une paroi rigide. En combinant toutes les observables, nous sommes capables d'atteindre des résolutions spatiales et de force de l'ordre du femtoNewton à l'échelle nanométrique, limitées par le bruit thermique. Ce travail a finalement fait l'objet d'une publication dans le journal \textit{Physical Review Research}, disponible en annexe \ref{app:prr}.

Comme indiqué dans le dernier chapitre, nous utilisons actuellement notre analyse statistique multi-échelle pour examiner la théorie de la lubrification douce appliquée au mouvement brownien. En particulier, si une particule brownienne peut déclencher un soulèvement élastohydrodynamique, qui est une force non conservative. Nous travaillons également à pousser la limite de notre résolution pour mesurer les effets fins sur les cumulents d'ordre élevé afin de vérifier expérimentalement un modèle théorique, qui présente une corrélation entre le déplacement perpendiculaire et parallèle d'une particule diffusant librement, ce qui peut être contre-intuitif et n'a pas encore été mesuré. Enfin, nous pensons que la capacité de mesurer de minuscules forces de surface, localement et à l'équilibre, ainsi que l'extension possible de la méthode aux forces non-conservatives et aux situations hors équilibre, ouvre des perspectives fascinantes pour la nanophysique et la biophysique.

