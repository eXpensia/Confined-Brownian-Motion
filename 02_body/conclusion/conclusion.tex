\section{Conclusion}

In this manuscript, we have covered a broad spectrum of the Brownian motion. We started from its history by presenting its first observation in the 19th by Robert Brown 

In this chapter, we have covered the history of Brownian motion, from the first observation by Robert Brown in the middle of the 19th century to its mathematical and experimental proofs in the early 20th century. We have then described mathematically the bulk Brownian motion and its important statistical properties. Finally, we have used the latter description to simulate Brownian motion using both the full-Langevin equation, and its over-damped version. 


In this chapter, we have covered different techniques that enable the tracking of individual microparticles. Each method has pros and cons. We decided to employ the Lorenz-Mie framework since it requires no calibration. Then, we have shown how we implement it in practice, from the experimental setup to the numerical treatment. An example of the Jupyter notebooks employed for the tracking can be found in appendix \ref{app:tracking}.  We have discussed how to have fast and accurate fits to retrieve the particle trajectory. To do so, we first characterize the particle fully, namely, its radius and optical index, analyzing solely the trajectory over before tracking a whole video. 

Now that we have an understanding on the tracking of single colloids, we can use the measured trajectories in order to understand how the Brownian motion is affected in various configurations. 

A given set of height, optical index and radius of a colloid thus gives unique holograms. Conversely, this uniqueness of the holograms permits precise extraction of the position, optical index and radius of a colloid. Holograms for different sets of parameters are shown in Figs.\ref{fig:holo_fix_z} and \ref{fig:holo_fix_n}. Additionally, the interested reader can use  the Jupyter Notebook on my Github repository in order to plot any hologram \href{https://github.com/eXpensia/Ma-these/blob/main/02_body/chapter2/images/holo_size_exemple/holosize_variation.ipynb}{\faGithub}.  

Finally, the Lorenz-Mie framework provides the most versatile in-line holographic method. Indeed, it permits tracking and characterizing unique particles even without a priori knowledge on its characteristics. Besides, it is possible to write the Lorenz-Mie function $\vec{f}_\mathrm{s}$ for particular cases such as anisotropic particles \cite{fung_holographic_2013, wang_using_2014} or particle clusters \cite{fung_holographic_2013, perry_real-space_2013} to name a few. Such possibilities pave the way to a lot of experimental studies. Additionally, the method allows reaching a really high precision as the tenth of nanometers on the position and radius as well as $10^{-3}$ on the optical index \cite{lee_characterizing_2007}. 

Unfortunately, the Lorenz-Mie framework suffers from a major drawback which is the time needed to fit one image. For example, a 200 by 200 pixels image, of a $2.5 ~ \mathrm{\mu m}$ particle's hologram, can take up to two minutes to be fitted using a pure and straightforward Python algorithm. A lot of work has been done to permit faster tracking, such as random-subset fitting \cite{dimiduk_random-subset_2014}, GPU (graphical processing units) acceleration, machine learning \cite{yevick_machine-learning_2014, hannel_machine-learning_2018} and deep neural networks \cite{altman_catch_2020}.

To conclude, we have successfully built a multi-scale statistical analysis for the problem of freely diffusing individual colloids near a rigid wall. Combining the equilibrium distribution in position, time-dependent non-Gaussian statistics for the spatial displacements, a novel method to infer local diffusion coefficients, and a multifitting procedure, allowed us to reduce drastically the measurement uncertainties and reach the nanoscale and thermal-noise-limited femtoNewton spatial and force resolutions, respectively. The ability to measure tiny surface forces, locally, and at equilibrium, as well the possible extension of the method to non-conservative forces and out-of-equilibrium settings~\cite{amarouchene_nonequilibrium_2019, mangeat_role_2019}, opens fascinating perspectives for nanophysics and biophysics.

