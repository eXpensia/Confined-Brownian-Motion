


















\section{Conclusion}

In this manuscript, we have addressed experimentally, numerically and theoretically several aspects related to  Brownian motion. In chapter \ref{sec:chapter1}, we started with history by presenting the first observation in the XIXth century by Robert Brown and the subsequent mathematical description, and numerical simulation. We have then, in chapter \ref{sec:chapter2}, reviewed different techniques that permit tracking of individual microparticles. We mainly focused on the Lorenz-Mie framework that we used in order to study the free confined Brownian motion. Indeed, it requires no calibration by offering a direct measure of the radius and optical index of each tracked particle. We then showed the experimental setup, which is a custom-made inverted microscope that we optimized over the years to capture Lorenz-Mie holograms. To retrieve the trajectory from the captured holograms we mainly used a tool developped at the Grier's Lab under the name of Pylorenzmie \href{https://github.com/davidgrier/pylorenzmie}{\faGithub} and a custom version, Wraplorenzmie, to automate the tracking across whole movies and simplify the use of MP4 files \href{https://github.com/eXpensia/wraplorenzmie}{\faGithub}.

The particles thus tracked and the trajectories retrieved, in chapter \ref{chap3}, we focused on the analysis of Brownian motion near a rigid and charged wall. We first detailed in section \ref{sec:confined} how the physics changes due to the wall, from the DLVO interactions between the surface and the particles to hindered diffusion. Due the latter, a spurious drift appears in the overdamped Fokker-Planck equation which needs to be taken into account in numerical simulations, and also for force measurements.

Once the underlying theory has been explained,in section \ref{section:expresults}, we described the trajectory analysis. We addressed static observables such as equilibrium distribution, and dynamic observables such as the Mean Squared Displacements (\gls{MSD}s) and displacement distributions. In particular, for the motion perpendicular to the wall, the \gls{MSD} exhibits a normal distribution at short time, and an equilibrium plateau at long time. Furthermore, the short-time displacement distribution exhibits non-Gaussian properties which is a direct signature of the hindered mobility induced by the rigid boundary.

Once the statistical properties of a confined colloid understood and correctly measured, we then focused on measuring the hindered diffusion coefficient. This measure was done using a novel method developped by Frishman and Ronceray \cite{frishman_learning_2020}, using information theory to infere the local mobility onto a basis of fit functions, leading to high accuracy near the surface (where there is less data). 

All the data, from the equilibrium distribution to the hindered mobility can be described by the three parameters of the system $B$, $\ell _\mathrm{D}$ and $\ell_\mathrm{B}$. We thus built a multi-fitting method, that permits to infere precisely these parameters by taking into account all the observables at once. From this method, we extracted precisely the equilibrium potential from which we computed the conservative forces. This force measuremement was successfully corroborated by an independent one based on the drifts of the trajectories.

All together, we are able to reach a thermal-noise-limited femtonewton force resolution as well as a spatial resolution at the nanoscale.

As shown in chapter \ref{chapfin}, we are currently using our device and method in order to investigate an original coupling between soft lubrication theory and Brownian motion. Also, we are working on pushing the limits of our resolution to measure fine effects on higher-order cumulents. Finally, we believe that the ability to measure tiny surface forces, locally, and at equilibrium, as well as the possible application of the method towards non-conservative forces and out-of-equilibrium settings, open fascinating perspectives for nanophysics and biophysics.



\newpage

\section{Conclusion en français}

Dans ce manuscrit, nous avons abordé expérimentalement, numériquement et théoriquement plusieurs aspects liés au mouvement Brownien. Dans le chapitre \ref{sec:chapter1}, nous avons commencé par l'histoire, en présentant la première observation au XIXème siècle par Robert Brown et la description mathématique ainsi que la simulation numérique qui ont suivi. Nous avons ensuite, dans le chapitre \ref{sec:chapter2}, passé en revue les différentes techniques qui permettent de suivre des microparticules individuelles. Nous nous sommes principalement concentrés sur le cadre Lorenz-Mie que nous avons utilisé afin d'étudier le mouvement brownien libre confiné. En effet, il ne nécessite aucune calibration en offrant une mesure directe du rayon et de l'indice optique de chaque particule suivie. Nous avons ensuite montré le dispositif expérimental, qui est un microscope inversé fait sur mesure que nous avons optimisé au fil des ans pour capturer des hologrammes de Lorenz-Mie. Pour extraire la trajectoire des hologrammes capturés, nous avons principalement utilisé un outil développé au laboratoire de Grier sous le nom de Pylorenzmie \href{https://github.com/davidgrier/pylorenzmie}{\faGithub} et une version personnalisée, Wraplorenzmie, pour automatiser le suivi sur des films entiers et simplifier l'utilisation des fichiers MP4 \href{https://github.com/eXpensia/wraplorenzmie}{\faGithub}.

Les particules ainsi suivies et les trajectoires extraites, dans le chapitre \ref{chap3}, nous nous sommes intéressés à l'analyse du mouvement brownien à proximité d'une paroi rigide et chargée. Nous avons d'abord détaillé dans la section \ref{sec:confined} comment la physique change à cause du mur, des interactions DLVO entre la surface et les particules à la diffusion entravée. En raison de cette dernière, une dérive parasite apparaît dans l'équation de Fokker-Planck suramortie qui doit être prise en compte dans les simulations numériques, ainsi que pour les mesures de force.

Une fois la théorie sous-jacente expliquée, dans la section \ref{section:expresults}, nous avons décrit les résultats expérimentaux et la manière dont les trajectoires ont été analysées. Nous avons abordé les observables statiques comme la distribution d'équilibre, et les observables dynamiques comme les déplacements quadratiques moyens (\gls{MSD}s) et les distributions de déplacement. En particulier, pour le mouvement perpendiculaire au mur, le \gls{MSD} présente une distribution normale à court terme, et une valeur d'équilibre en plateau à long terme. De plus, la distribution du déplacement à court terme présente des propriétés non gaussiennes, ce qui est une signature directe de la modification de la mobilité induite par la frontière rigide.

Les propriétés statistiques d'un colloïde confiné étant comprises et correctement mesurées, nous nous sommes ensuite attachés à mesurer le coefficient de diffusion le coefficient de diffusion. Cette mesure a été effectuée à l'aide d'une nouvelle méthode développée par Frishman et Ronceray \cite{frishman_learning_2020}, utilisant la théorie de l'information pour déduire la mobilité locale sur une base de fonctions d'ajustement, conduisant à une grande précision près de la surface (où il y a moins de données). 

Toutes les données, de la distribution à l'équilibre à la mobilité, peuvent être décrites par les trois paramètres du système $B$, $\ell _\mathrm{D}$ et $\ell_\mathrm{B}$. Nous avons donc construit une méthode multi-fitting, qui permet de déduire précisément ces paramètres en prenant en compte toutes les observables à la fois. De cette méthode, nous avons extrait précisément le potentiel d'équilibre à partir duquel nous avons calculé les forces conservatives. Cette mesure de force a été corroborée avec succès par une mesure indépendante basée sur les dérives des trajectoires.

Au total, nous sommes capables d'atteindre une résolution au femtonewton sur la mesure de la force, limitée par le bruit thermique ainsi qu'une résolution spatiale à l'échelle nanométrique.

Comme le montre le chapitre \ref{chapfin}, nous utilisons actuellement notre dispositif et notre méthode afin d'étudier le couplage original entre la théorie de la lubrification douce et le mouvement brownien. Nous travaillons également à repousser les limites de notre résolution pour mesurer les effets fins sur les cumulants d'ordre supérieur. Enfin, nous pensons que la capacité à mesurer de minuscules forces de surface, localement et à l'équilibre, ainsi que l'application possible de la méthode aux forces non-conservatives et aux situations hors équilibre, ouvrent des perspectives fascinantes pour la nanophysique et la biophysique.


