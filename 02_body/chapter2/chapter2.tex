\newpage
\section{Particle characterization and particle tracking using holographic video microscopy}
		\label{sec:chapter2}

\subsection{Introduction}


Holographic video microscopy exists for a certain time, and a lot of different type of exists and have all their type of good and bad things, there is since 2005, we have Rayleigh-Sommerfeld back progapation and lorenz-mie fitting. The first one is great to track lots of particle at the same time and the latter is great to fit only one particle with unprecedent precision but is quite long to compute. 
In order to study find deviation of the statistical properties of the a free diffusing particle we do need to use the one which is more precise so we will use Lorenz mie tracking. In order to study it 

\subsection{In-line holographic video microscopy theory}

\subsubsection{Rayleigh sommerfeld back propagation}
je vais expliquer ici qu'est c'est queu le rauleigh sommerfeld back progagation et expliquer les limitations de cette méthode pour les travaux réaliser lors de ma thèse
\subsubsection{Lorenz-Mie fitting}
Je vais ici expliquer la théorie derrière l'inline holoogrpahy video microcopy.

\subsection{Experimental setup}
J'explique ici le setup experimental et comment il faut bien choisir son objectif en side note !!

\subsection{Numerical treatment ??}

Ici je vais présenter comment j'inverse les holograme, je vais présenter holopy et le code de groupe de grier. Expliquer aussi les limitations temporelles et comment j'ai essayer de les contourner pour finalement dire qu'un doctorant de chez grier a fini de dévelloper l'accélération GPU qui a rendu l'inverversage des trajectoire a un temps humain !

\subsection{Radius and optical index characterization}


Ici je vais présenter la characterization et les graphs de n en fonction de r et de discuter la dépendence apparente de r/n où je vais certainement m'appuyer sur un des papiers de Grier. 

