\newpage
\section{Particle characterization and particle tracking using holographic video microscopy}
		\label{sec:chapter2}

\subsection{Introduction}

Properties of coherent light to produce interference is widely used in metrology for a long time with the famous Fabry-Pérot  \cite{fabry_theorie_1899, perot_application_1899} or Michelson interferometer \cite{michelson_relative_1887} which was initially used to measure the earth rotation and is still used today, in particular, for the recent measurement of gravitational waves
\cite{ligo_scientific_collaboration_and_virgo_collaboration_gw151226_2016}. 
Since the beginning of the century, interest on tracking and characterizing colloidal particles risen thanks to the democratization of micro fluidics and lab-on-a-chip technologies. A lot of methods were developed, I will in the folowing give some insights on the three most used :

\begin{itemize}
	\item Reflection Interference Contrast Microscopy (\gls{RICM})
	\item Rayleigh-Sommerfeld back-propagation
	\item Lorenz-Mie fit
\end{itemize}

Those methods use directly the images captured from a microscope without any further treatments, thus, they are generally called in-line holographic video microscopy.

\subsection{In-line holographic video microscopy theory}

\subsubsection{Reflection Interference Contrast Microscopy}


Reflection Interference Contrast Microscopy was first introduced in cell biology by Curtis to study embryonic chick heart fibroblast \cite{curtis_mechanism_1964} in 1964. \gls{RICM} gained in popularity 40 years after in the same field \cite{filler_reflection_2000, siver_use_2000, weber_2_2003, limozin_quantitative_2009} and was recently used in soft matter physics to study elastohydrodynamic lift at a soft wall \cite{davies_elastohydrodynamic_2018}.

When we illuminate a colloid with a plane wave, a part of the light is reflected from the surface and interference fringes appear. Let's take an interest at the mathematical description of this phenomenon. In the far field, we can describe two different one-dimensional electric field vectors of the same pulsation $\omega$ \cite{f_bohren_absorption_1998} as:

\begin{equation}
	\vec{E_1}(r, t) = \vec{E_{0_1}} \cos(k_1r - \omega t + \epsilon_1) ~,
\end{equation}
and
\begin{equation}
	\vec{E_2}(r, t) = \vec{E_{0_2}} \cos (k_2r - \omega t + \epsilon_1) ~.
\end{equation}


\nomenclature{$\vec{E}$}{Electrical field}
\nomenclature{$k$}{Wave number}
\nomenclature{$\omega$}{Pulsation}

Where the $k$ is the wave number $k=2\pi/\lambda$, $\lambda$ denoting the wavelength, $\epsilon$ the initial phase of each waves and $r$ the position from the source.

\subsubsection{Rayleigh sommerfeld back propagation}
je vais expliquer ici qu'est c'est queu le rauleigh sommerfeld back progagation et expliquer les limitations de cette méthode pour les travaux réaliser lors de ma thèse
\subsubsection{Lorenz-Mie fitting}
Je vais ici expliquer la théorie derrière l'inline holoogrpahy video microcopy.

\subsection{Experimental setup}
J'explique ici le setup experimental et comment il faut bien choisir son objectif en side note !!

\subsection{Numerical treatment ??}

Ici je vais présenter comment j'inverse les holograme, je vais présenter holopy et le code de groupe de grier. Expliquer aussi les limitations temporelles et comment j'ai essayer de les contourner pour finalement dire qu'un doctorant de chez grier a fini de dévelloper l'accélération GPU qui a rendu l'inverversage des trajectoire a un temps humain !

\subsection{Radius and optical index characterization}


Ici je vais présenter la characterization et les graphs de n en fonction de r et de discuter la dépendence apparente de r/n où je vais certainement m'appuyer sur un des papiers de Grier. 

