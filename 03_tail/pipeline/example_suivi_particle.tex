\documentclass[12pt]{article}

    \usepackage[breakable]{tcolorbox}
    \usepackage{parskip} % Stop auto-indenting (to mimic markdown behaviour)
    
    \usepackage{iftex}
    \ifPDFTeX
    	\usepackage[T1]{fontenc}
    	\usepackage{mathpazo}
    \else
    	\usepackage{fontspec}
    \fi

    % Basic figure setup, for now with no caption control since it's done
    % automatically by Pandoc (which extracts ![](path) syntax from Markdown).
    \usepackage{graphicx}
    % Maintain compatibility with old templates. Remove in nbconvert 6.0
    \let\Oldincludegraphics\includegraphics
    % Ensure that by default, figures have no caption (until we provide a
    % proper Figure object with a Caption API and a way to capture that
    % in the conversion process - todo).
    \usepackage{caption}
    \DeclareCaptionFormat{nocaption}{}
    \captionsetup{format=nocaption,aboveskip=0pt,belowskip=0pt}

    \usepackage{float}
    \floatplacement{figure}{H} % forces figures to be placed at the correct location
    \usepackage{xcolor} % Allow colors to be defined
    \usepackage{enumerate} % Needed for markdown enumerations to work
    \usepackage{geometry} % Used to adjust the document margins
    \usepackage{amsmath} % Equations
    \usepackage{amssymb} % Equations
    \usepackage{textcomp} % defines textquotesingle
    % Hack from http://tex.stackexchange.com/a/47451/13684:
    \AtBeginDocument{%
        \def\PYZsq{\textquotesingle}% Upright quotes in Pygmentized code
    }
    \usepackage{upquote} % Upright quotes for verbatim code
    \usepackage{eurosym} % defines \euro
    \usepackage[mathletters]{ucs} % Extended unicode (utf-8) support
    \usepackage{fancyvrb} % verbatim replacement that allows latex
    \usepackage{grffile} % extends the file name processing of package graphics 
                         % to support a larger range
    \makeatletter % fix for old versions of grffile with XeLaTeX
    \@ifpackagelater{grffile}{2019/11/01}
    {
      % Do nothing on new versions
    }
    {
      \def\Gread@@xetex#1{%
        \IfFileExists{"\Gin@base".bb}%
        {\Gread@eps{\Gin@base.bb}}%
        {\Gread@@xetex@aux#1}%
      }
    }
    \makeatother
    \usepackage[Export]{adjustbox} % Used to constrain images to a maximum size
    \adjustboxset{max size={0.9\linewidth}{0.9\paperheight}}

    % The hyperref package gives us a pdf with properly built
    % internal navigation ('pdf bookmarks' for the table of contents,
    % internal cross-reference links, web links for URLs, etc.)
    \usepackage{hyperref}
    % The default LaTeX title has an obnoxious amount of whitespace. By default,
    % titling removes some of it. It also provides customization options.
    \usepackage{titling}
    \usepackage{longtable} % longtable support required by pandoc >1.10
    \usepackage{booktabs}  % table support for pandoc > 1.12.2
    \usepackage[inline]{enumitem} % IRkernel/repr support (it uses the enumerate* environment)
    \usepackage[normalem]{ulem} % ulem is needed to support strikethroughs (\sout)
                                % normalem makes italics be italics, not underlines
    \usepackage{mathrsfs}
    

    
    % Colors for the hyperref package
    \definecolor{urlcolor}{rgb}{0,.145,.698}
    \definecolor{linkcolor}{rgb}{.71,0.21,0.01}
    \definecolor{citecolor}{rgb}{.12,.54,.11}

    % ANSI colors
    \definecolor{ansi-black}{HTML}{3E424D}
    \definecolor{ansi-black-intense}{HTML}{282C36}
    \definecolor{ansi-red}{HTML}{E75C58}
    \definecolor{ansi-red-intense}{HTML}{B22B31}
    \definecolor{ansi-green}{HTML}{00A250}
    \definecolor{ansi-green-intense}{HTML}{007427}
    \definecolor{ansi-yellow}{HTML}{DDB62B}
    \definecolor{ansi-yellow-intense}{HTML}{B27D12}
    \definecolor{ansi-blue}{HTML}{208FFB}
    \definecolor{ansi-blue-intense}{HTML}{0065CA}
    \definecolor{ansi-magenta}{HTML}{D160C4}
    \definecolor{ansi-magenta-intense}{HTML}{A03196}
    \definecolor{ansi-cyan}{HTML}{60C6C8}
    \definecolor{ansi-cyan-intense}{HTML}{258F8F}
    \definecolor{ansi-white}{HTML}{C5C1B4}
    \definecolor{ansi-white-intense}{HTML}{A1A6B2}
    \definecolor{ansi-default-inverse-fg}{HTML}{FFFFFF}
    \definecolor{ansi-default-inverse-bg}{HTML}{000000}

    % common color for the border for error outputs.
    \definecolor{outerrorbackground}{HTML}{FFDFDF}

    % commands and environments needed by pandoc snippets
    % extracted from the output of `pandoc -s`
    \providecommand{\tightlist}{%
      \setlength{\itemsep}{0pt}\setlength{\parskip}{0pt}}
    \DefineVerbatimEnvironment{Highlighting}{Verbatim}{commandchars=\\\{\}}
    % Add ',fontsize=\small' for more characters per line
    \newenvironment{Shaded}{}{}
    \newcommand{\KeywordTok}[1]{\textcolor[rgb]{0.00,0.44,0.13}{\textbf{{#1}}}}
    \newcommand{\DataTypeTok}[1]{\textcolor[rgb]{0.56,0.13,0.00}{{#1}}}
    \newcommand{\DecValTok}[1]{\textcolor[rgb]{0.25,0.63,0.44}{{#1}}}
    \newcommand{\BaseNTok}[1]{\textcolor[rgb]{0.25,0.63,0.44}{{#1}}}
    \newcommand{\FloatTok}[1]{\textcolor[rgb]{0.25,0.63,0.44}{{#1}}}
    \newcommand{\CharTok}[1]{\textcolor[rgb]{0.25,0.44,0.63}{{#1}}}
    \newcommand{\StringTok}[1]{\textcolor[rgb]{0.25,0.44,0.63}{{#1}}}
    \newcommand{\CommentTok}[1]{\textcolor[rgb]{0.38,0.63,0.69}{\textit{{#1}}}}
    \newcommand{\OtherTok}[1]{\textcolor[rgb]{0.00,0.44,0.13}{{#1}}}
    \newcommand{\AlertTok}[1]{\textcolor[rgb]{1.00,0.00,0.00}{\textbf{{#1}}}}
    \newcommand{\FunctionTok}[1]{\textcolor[rgb]{0.02,0.16,0.49}{{#1}}}
    \newcommand{\RegionMarkerTok}[1]{{#1}}
    \newcommand{\ErrorTok}[1]{\textcolor[rgb]{1.00,0.00,0.00}{\textbf{{#1}}}}
    \newcommand{\NormalTok}[1]{{#1}}
    
    % Additional commands for more recent versions of Pandoc
    \newcommand{\ConstantTok}[1]{\textcolor[rgb]{0.53,0.00,0.00}{{#1}}}
    \newcommand{\SpecialCharTok}[1]{\textcolor[rgb]{0.25,0.44,0.63}{{#1}}}
    \newcommand{\VerbatimStringTok}[1]{\textcolor[rgb]{0.25,0.44,0.63}{{#1}}}
    \newcommand{\SpecialStringTok}[1]{\textcolor[rgb]{0.73,0.40,0.53}{{#1}}}
    \newcommand{\ImportTok}[1]{{#1}}
    \newcommand{\DocumentationTok}[1]{\textcolor[rgb]{0.73,0.13,0.13}{\textit{{#1}}}}
    \newcommand{\AnnotationTok}[1]{\textcolor[rgb]{0.38,0.63,0.69}{\textbf{\textit{{#1}}}}}
    \newcommand{\CommentVarTok}[1]{\textcolor[rgb]{0.38,0.63,0.69}{\textbf{\textit{{#1}}}}}
    \newcommand{\VariableTok}[1]{\textcolor[rgb]{0.10,0.09,0.49}{{#1}}}
    \newcommand{\ControlFlowTok}[1]{\textcolor[rgb]{0.00,0.44,0.13}{\textbf{{#1}}}}
    \newcommand{\OperatorTok}[1]{\textcolor[rgb]{0.40,0.40,0.40}{{#1}}}
    \newcommand{\BuiltInTok}[1]{{#1}}
    \newcommand{\ExtensionTok}[1]{{#1}}
    \newcommand{\PreprocessorTok}[1]{\textcolor[rgb]{0.74,0.48,0.00}{{#1}}}
    \newcommand{\AttributeTok}[1]{\textcolor[rgb]{0.49,0.56,0.16}{{#1}}}
    \newcommand{\InformationTok}[1]{\textcolor[rgb]{0.38,0.63,0.69}{\textbf{\textit{{#1}}}}}
    \newcommand{\WarningTok}[1]{\textcolor[rgb]{0.38,0.63,0.69}{\textbf{\textit{{#1}}}}}
    
    
    % Define a nice break command that doesn't care if a line doesn't already
    % exist.
    \def\br{\hspace*{\fill} \\* }
    % Math Jax compatibility definitions
    \def\gt{>}
    \def\lt{<}
    \let\Oldtex\TeX
    \let\Oldlatex\LaTeX
    \renewcommand{\TeX}{\textrm{\Oldtex}}
    \renewcommand{\LaTeX}{\textrm{\Oldlatex}}
    % Document parameters
    % Document title
    \title{example\_suivi\_particle}
    
    
    
    
    
% Pygments definitions
\makeatletter
\def\PY@reset{\let\PY@it=\relax \let\PY@bf=\relax%
    \let\PY@ul=\relax \let\PY@tc=\relax%
    \let\PY@bc=\relax \let\PY@ff=\relax}
\def\PY@tok#1{\csname PY@tok@#1\endcsname}
\def\PY@toks#1+{\ifx\relax#1\empty\else%
    \PY@tok{#1}\expandafter\PY@toks\fi}
\def\PY@do#1{\PY@bc{\PY@tc{\PY@ul{%
    \PY@it{\PY@bf{\PY@ff{#1}}}}}}}
\def\PY#1#2{\PY@reset\PY@toks#1+\relax+\PY@do{#2}}

\@namedef{PY@tok@w}{\def\PY@tc##1{\textcolor[rgb]{0.73,0.73,0.73}{##1}}}
\@namedef{PY@tok@c}{\let\PY@it=\textit\def\PY@tc##1{\textcolor[rgb]{0.25,0.50,0.50}{##1}}}
\@namedef{PY@tok@cp}{\def\PY@tc##1{\textcolor[rgb]{0.74,0.48,0.00}{##1}}}
\@namedef{PY@tok@k}{\let\PY@bf=\textbf\def\PY@tc##1{\textcolor[rgb]{0.00,0.50,0.00}{##1}}}
\@namedef{PY@tok@kp}{\def\PY@tc##1{\textcolor[rgb]{0.00,0.50,0.00}{##1}}}
\@namedef{PY@tok@kt}{\def\PY@tc##1{\textcolor[rgb]{0.69,0.00,0.25}{##1}}}
\@namedef{PY@tok@o}{\def\PY@tc##1{\textcolor[rgb]{0.40,0.40,0.40}{##1}}}
\@namedef{PY@tok@ow}{\let\PY@bf=\textbf\def\PY@tc##1{\textcolor[rgb]{0.67,0.13,1.00}{##1}}}
\@namedef{PY@tok@nb}{\def\PY@tc##1{\textcolor[rgb]{0.00,0.50,0.00}{##1}}}
\@namedef{PY@tok@nf}{\def\PY@tc##1{\textcolor[rgb]{0.00,0.00,1.00}{##1}}}
\@namedef{PY@tok@nc}{\let\PY@bf=\textbf\def\PY@tc##1{\textcolor[rgb]{0.00,0.00,1.00}{##1}}}
\@namedef{PY@tok@nn}{\let\PY@bf=\textbf\def\PY@tc##1{\textcolor[rgb]{0.00,0.00,1.00}{##1}}}
\@namedef{PY@tok@ne}{\let\PY@bf=\textbf\def\PY@tc##1{\textcolor[rgb]{0.82,0.25,0.23}{##1}}}
\@namedef{PY@tok@nv}{\def\PY@tc##1{\textcolor[rgb]{0.10,0.09,0.49}{##1}}}
\@namedef{PY@tok@no}{\def\PY@tc##1{\textcolor[rgb]{0.53,0.00,0.00}{##1}}}
\@namedef{PY@tok@nl}{\def\PY@tc##1{\textcolor[rgb]{0.63,0.63,0.00}{##1}}}
\@namedef{PY@tok@ni}{\let\PY@bf=\textbf\def\PY@tc##1{\textcolor[rgb]{0.60,0.60,0.60}{##1}}}
\@namedef{PY@tok@na}{\def\PY@tc##1{\textcolor[rgb]{0.49,0.56,0.16}{##1}}}
\@namedef{PY@tok@nt}{\let\PY@bf=\textbf\def\PY@tc##1{\textcolor[rgb]{0.00,0.50,0.00}{##1}}}
\@namedef{PY@tok@nd}{\def\PY@tc##1{\textcolor[rgb]{0.67,0.13,1.00}{##1}}}
\@namedef{PY@tok@s}{\def\PY@tc##1{\textcolor[rgb]{0.73,0.13,0.13}{##1}}}
\@namedef{PY@tok@sd}{\let\PY@it=\textit\def\PY@tc##1{\textcolor[rgb]{0.73,0.13,0.13}{##1}}}
\@namedef{PY@tok@si}{\let\PY@bf=\textbf\def\PY@tc##1{\textcolor[rgb]{0.73,0.40,0.53}{##1}}}
\@namedef{PY@tok@se}{\let\PY@bf=\textbf\def\PY@tc##1{\textcolor[rgb]{0.73,0.40,0.13}{##1}}}
\@namedef{PY@tok@sr}{\def\PY@tc##1{\textcolor[rgb]{0.73,0.40,0.53}{##1}}}
\@namedef{PY@tok@ss}{\def\PY@tc##1{\textcolor[rgb]{0.10,0.09,0.49}{##1}}}
\@namedef{PY@tok@sx}{\def\PY@tc##1{\textcolor[rgb]{0.00,0.50,0.00}{##1}}}
\@namedef{PY@tok@m}{\def\PY@tc##1{\textcolor[rgb]{0.40,0.40,0.40}{##1}}}
\@namedef{PY@tok@gh}{\let\PY@bf=\textbf\def\PY@tc##1{\textcolor[rgb]{0.00,0.00,0.50}{##1}}}
\@namedef{PY@tok@gu}{\let\PY@bf=\textbf\def\PY@tc##1{\textcolor[rgb]{0.50,0.00,0.50}{##1}}}
\@namedef{PY@tok@gd}{\def\PY@tc##1{\textcolor[rgb]{0.63,0.00,0.00}{##1}}}
\@namedef{PY@tok@gi}{\def\PY@tc##1{\textcolor[rgb]{0.00,0.63,0.00}{##1}}}
\@namedef{PY@tok@gr}{\def\PY@tc##1{\textcolor[rgb]{1.00,0.00,0.00}{##1}}}
\@namedef{PY@tok@ge}{\let\PY@it=\textit}
\@namedef{PY@tok@gs}{\let\PY@bf=\textbf}
\@namedef{PY@tok@gp}{\let\PY@bf=\textbf\def\PY@tc##1{\textcolor[rgb]{0.00,0.00,0.50}{##1}}}
\@namedef{PY@tok@go}{\def\PY@tc##1{\textcolor[rgb]{0.53,0.53,0.53}{##1}}}
\@namedef{PY@tok@gt}{\def\PY@tc##1{\textcolor[rgb]{0.00,0.27,0.87}{##1}}}
\@namedef{PY@tok@err}{\def\PY@bc##1{{\setlength{\fboxsep}{\string -\fboxrule}\fcolorbox[rgb]{1.00,0.00,0.00}{1,1,1}{\strut ##1}}}}
\@namedef{PY@tok@kc}{\let\PY@bf=\textbf\def\PY@tc##1{\textcolor[rgb]{0.00,0.50,0.00}{##1}}}
\@namedef{PY@tok@kd}{\let\PY@bf=\textbf\def\PY@tc##1{\textcolor[rgb]{0.00,0.50,0.00}{##1}}}
\@namedef{PY@tok@kn}{\let\PY@bf=\textbf\def\PY@tc##1{\textcolor[rgb]{0.00,0.50,0.00}{##1}}}
\@namedef{PY@tok@kr}{\let\PY@bf=\textbf\def\PY@tc##1{\textcolor[rgb]{0.00,0.50,0.00}{##1}}}
\@namedef{PY@tok@bp}{\def\PY@tc##1{\textcolor[rgb]{0.00,0.50,0.00}{##1}}}
\@namedef{PY@tok@fm}{\def\PY@tc##1{\textcolor[rgb]{0.00,0.00,1.00}{##1}}}
\@namedef{PY@tok@vc}{\def\PY@tc##1{\textcolor[rgb]{0.10,0.09,0.49}{##1}}}
\@namedef{PY@tok@vg}{\def\PY@tc##1{\textcolor[rgb]{0.10,0.09,0.49}{##1}}}
\@namedef{PY@tok@vi}{\def\PY@tc##1{\textcolor[rgb]{0.10,0.09,0.49}{##1}}}
\@namedef{PY@tok@vm}{\def\PY@tc##1{\textcolor[rgb]{0.10,0.09,0.49}{##1}}}
\@namedef{PY@tok@sa}{\def\PY@tc##1{\textcolor[rgb]{0.73,0.13,0.13}{##1}}}
\@namedef{PY@tok@sb}{\def\PY@tc##1{\textcolor[rgb]{0.73,0.13,0.13}{##1}}}
\@namedef{PY@tok@sc}{\def\PY@tc##1{\textcolor[rgb]{0.73,0.13,0.13}{##1}}}
\@namedef{PY@tok@dl}{\def\PY@tc##1{\textcolor[rgb]{0.73,0.13,0.13}{##1}}}
\@namedef{PY@tok@s2}{\def\PY@tc##1{\textcolor[rgb]{0.73,0.13,0.13}{##1}}}
\@namedef{PY@tok@sh}{\def\PY@tc##1{\textcolor[rgb]{0.73,0.13,0.13}{##1}}}
\@namedef{PY@tok@s1}{\def\PY@tc##1{\textcolor[rgb]{0.73,0.13,0.13}{##1}}}
\@namedef{PY@tok@mb}{\def\PY@tc##1{\textcolor[rgb]{0.40,0.40,0.40}{##1}}}
\@namedef{PY@tok@mf}{\def\PY@tc##1{\textcolor[rgb]{0.40,0.40,0.40}{##1}}}
\@namedef{PY@tok@mh}{\def\PY@tc##1{\textcolor[rgb]{0.40,0.40,0.40}{##1}}}
\@namedef{PY@tok@mi}{\def\PY@tc##1{\textcolor[rgb]{0.40,0.40,0.40}{##1}}}
\@namedef{PY@tok@il}{\def\PY@tc##1{\textcolor[rgb]{0.40,0.40,0.40}{##1}}}
\@namedef{PY@tok@mo}{\def\PY@tc##1{\textcolor[rgb]{0.40,0.40,0.40}{##1}}}
\@namedef{PY@tok@ch}{\let\PY@it=\textit\def\PY@tc##1{\textcolor[rgb]{0.25,0.50,0.50}{##1}}}
\@namedef{PY@tok@cm}{\let\PY@it=\textit\def\PY@tc##1{\textcolor[rgb]{0.25,0.50,0.50}{##1}}}
\@namedef{PY@tok@cpf}{\let\PY@it=\textit\def\PY@tc##1{\textcolor[rgb]{0.25,0.50,0.50}{##1}}}
\@namedef{PY@tok@c1}{\let\PY@it=\textit\def\PY@tc##1{\textcolor[rgb]{0.25,0.50,0.50}{##1}}}
\@namedef{PY@tok@cs}{\let\PY@it=\textit\def\PY@tc##1{\textcolor[rgb]{0.25,0.50,0.50}{##1}}}

\def\PYZbs{\char`\\}
\def\PYZus{\char`\_}
\def\PYZob{\char`\{}
\def\PYZcb{\char`\}}
\def\PYZca{\char`\^}
\def\PYZam{\char`\&}
\def\PYZlt{\char`\<}
\def\PYZgt{\char`\>}
\def\PYZsh{\char`\#}
\def\PYZpc{\char`\%}
\def\PYZdl{\char`\$}
\def\PYZhy{\char`\-}
\def\PYZsq{\char`\'}
\def\PYZdq{\char`\"}
\def\PYZti{\char`\~}
% for compatibility with earlier versions
\def\PYZat{@}
\def\PYZlb{[}
\def\PYZrb{]}
\makeatother


    % For linebreaks inside Verbatim environment from package fancyvrb. 
    \makeatletter
        \newbox\Wrappedcontinuationbox 
        \newbox\Wrappedvisiblespacebox 
        \newcommand*\Wrappedvisiblespace {\textcolor{red}{\textvisiblespace}} 
        \newcommand*\Wrappedcontinuationsymbol {\textcolor{red}{\llap{\tiny$\m@th\hookrightarrow$}}} 
        \newcommand*\Wrappedcontinuationindent {3ex } 
        \newcommand*\Wrappedafterbreak {\kern\Wrappedcontinuationindent\copy\Wrappedcontinuationbox} 
        % Take advantage of the already applied Pygments mark-up to insert 
        % potential linebreaks for TeX processing. 
        %        {, <, #, %, $, ' and ": go to next line. 
        %        _, }, ^, &, >, - and ~: stay at end of broken line. 
        % Use of \textquotesingle for straight quote. 
        \newcommand*\Wrappedbreaksatspecials {% 
            \def\PYGZus{\discretionary{\char`\_}{\Wrappedafterbreak}{\char`\_}}% 
            \def\PYGZob{\discretionary{}{\Wrappedafterbreak\char`\{}{\char`\{}}% 
            \def\PYGZcb{\discretionary{\char`\}}{\Wrappedafterbreak}{\char`\}}}% 
            \def\PYGZca{\discretionary{\char`\^}{\Wrappedafterbreak}{\char`\^}}% 
            \def\PYGZam{\discretionary{\char`\&}{\Wrappedafterbreak}{\char`\&}}% 
            \def\PYGZlt{\discretionary{}{\Wrappedafterbreak\char`\<}{\char`\<}}% 
            \def\PYGZgt{\discretionary{\char`\>}{\Wrappedafterbreak}{\char`\>}}% 
            \def\PYGZsh{\discretionary{}{\Wrappedafterbreak\char`\#}{\char`\#}}% 
            \def\PYGZpc{\discretionary{}{\Wrappedafterbreak\char`\%}{\char`\%}}% 
            \def\PYGZdl{\discretionary{}{\Wrappedafterbreak\char`\$}{\char`\$}}% 
            \def\PYGZhy{\discretionary{\char`\-}{\Wrappedafterbreak}{\char`\-}}% 
            \def\PYGZsq{\discretionary{}{\Wrappedafterbreak\textquotesingle}{\textquotesingle}}% 
            \def\PYGZdq{\discretionary{}{\Wrappedafterbreak\char`\"}{\char`\"}}% 
            \def\PYGZti{\discretionary{\char`\~}{\Wrappedafterbreak}{\char`\~}}% 
        } 
        % Some characters . , ; ? ! / are not pygmentized. 
        % This macro makes them "active" and they will insert potential linebreaks 
        \newcommand*\Wrappedbreaksatpunct {% 
            \lccode`\~`\.\lowercase{\def~}{\discretionary{\hbox{\char`\.}}{\Wrappedafterbreak}{\hbox{\char`\.}}}% 
            \lccode`\~`\,\lowercase{\def~}{\discretionary{\hbox{\char`\,}}{\Wrappedafterbreak}{\hbox{\char`\,}}}% 
            \lccode`\~`\;\lowercase{\def~}{\discretionary{\hbox{\char`\;}}{\Wrappedafterbreak}{\hbox{\char`\;}}}% 
            \lccode`\~`\:\lowercase{\def~}{\discretionary{\hbox{\char`\:}}{\Wrappedafterbreak}{\hbox{\char`\:}}}% 
            \lccode`\~`\?\lowercase{\def~}{\discretionary{\hbox{\char`\?}}{\Wrappedafterbreak}{\hbox{\char`\?}}}% 
            \lccode`\~`\!\lowercase{\def~}{\discretionary{\hbox{\char`\!}}{\Wrappedafterbreak}{\hbox{\char`\!}}}% 
            \lccode`\~`\/\lowercase{\def~}{\discretionary{\hbox{\char`\/}}{\Wrappedafterbreak}{\hbox{\char`\/}}}% 
            \catcode`\.\active
            \catcode`\,\active 
            \catcode`\;\active
            \catcode`\:\active
            \catcode`\?\active
            \catcode`\!\active
            \catcode`\/\active 
            \lccode`\~`\~ 	
        }
    \makeatother

    \let\OriginalVerbatim=\Verbatim
    \makeatletter
    \renewcommand{\Verbatim}[1][1]{%
        %\parskip\z@skip
        \sbox\Wrappedcontinuationbox {\Wrappedcontinuationsymbol}%
        \sbox\Wrappedvisiblespacebox {\FV@SetupFont\Wrappedvisiblespace}%
        \def\FancyVerbFormatLine ##1{\hsize\linewidth
            \vtop{\raggedright\hyphenpenalty\z@\exhyphenpenalty\z@
                \doublehyphendemerits\z@\finalhyphendemerits\z@
                \strut ##1\strut}%
        }%
        % If the linebreak is at a space, the latter will be displayed as visible
        % space at end of first line, and a continuation symbol starts next line.
        % Stretch/shrink are however usually zero for typewriter font.
        \def\FV@Space {%
            \nobreak\hskip\z@ plus\fontdimen3\font minus\fontdimen4\font
            \discretionary{\copy\Wrappedvisiblespacebox}{\Wrappedafterbreak}
            {\kern\fontdimen2\font}%
        }%
        
        % Allow breaks at special characters using \PYG... macros.
        \Wrappedbreaksatspecials
        % Breaks at punctuation characters . , ; ? ! and / need catcode=\active 	
        \OriginalVerbatim[#1,codes*=\Wrappedbreaksatpunct]%
    }
    \makeatother

    % Exact colors from NB
    \definecolor{incolor}{HTML}{303F9F}
    \definecolor{outcolor}{HTML}{D84315}
    \definecolor{cellborder}{HTML}{CFCFCF}
    \definecolor{cellbackground}{HTML}{F7F7F7}
    
    % prompt
    \makeatletter
    \newcommand{\boxspacing}{\kern\kvtcb@left@rule\kern\kvtcb@boxsep}
    \makeatother
    \newcommand{\prompt}[4]{
        {\ttfamily\llap{{\color{#2}[#3]:\hspace{3pt}#4}}\vspace{-\baselineskip}}
    }
    

    
    % Prevent overflowing lines due to hard-to-break entities
    \sloppy 
    % Setup hyperref package
    \hypersetup{
      breaklinks=true,  % so long urls are correctly broken across lines
      colorlinks=true,
      urlcolor=urlcolor,
      linkcolor=linkcolor,
      citecolor=citecolor,
      }
    % Slightly bigger margins than the latex defaults
    
    \geometry{verbose,tmargin=1in,bmargin=1in,lmargin=1in,rmargin=1in}
    
    

\begin{document}
    
    
    

    
    \hypertarget{fitting-pipeline-using-pylorenzmie}{%
\section{Fitting pipeline using
pylorenzmie}\label{fitting-pipeline-using-pylorenzmie}}

In order to fit an hologram, I used the pylorenzmie model which provides
a set of python classes in order to analyse holographic microscopy data.

Pylorenzmie can be download on the David Grier's github repository:
https://github.com/davidgrier/pylorenzmie.

What I actually get from the experiments are mp4 movies, in order to
analyze them easily, I constructed a wrapper around the pylorenzmie
module which can be found on my repository:
https://github.com/eXpensia/wraplorenzmie.

This wrapper permits to do the following pipeline:

\begin{itemize}
\tightlist
\item
  Directly load the movies
\item
  Compute the back ground.
\item
  Use the first image in order to get the pre guesses
\item
  Fit the 10 000 first images to determine precisely the radius and
  index of a particle.
\item
  Use the later information in order to fit the whole movie (and save
  the data in the same time)
\end{itemize}

One that done, the trajectory be analyzed separately.

    \begin{tcolorbox}[breakable, size=fbox, boxrule=1pt, pad at break*=1mm,colback=cellbackground, colframe=cellborder]
\prompt{In}{incolor}{1}{\boxspacing}
\begin{Verbatim}[commandchars=\\\{\}]
\PY{c+c1}{\PYZsh{} We first start by import the important modules}

\PY{k+kn}{import} \PY{n+nn}{wraplorenzmie}\PY{n+nn}{.}\PY{n+nn}{utilities}\PY{n+nn}{.}\PY{n+nn}{utilities} \PY{k}{as} \PY{n+nn}{utilities}
\PY{k+kn}{import} \PY{n+nn}{wraplorenzmie}\PY{n+nn}{.}\PY{n+nn}{fits}\PY{n+nn}{.}\PY{n+nn}{fit} \PY{k}{as} \PY{n+nn}{fit}
\PY{k+kn}{import} \PY{n+nn}{imageio}
\PY{c+c1}{\PYZsh{} For Plotting.}
\PY{k+kn}{import} \PY{n+nn}{matplotlib}\PY{n+nn}{.}\PY{n+nn}{pyplot} \PY{k}{as} \PY{n+nn}{plt}
\PY{k+kn}{import} \PY{n+nn}{seaborn} \PY{k}{as} \PY{n+nn}{sns}
\PY{k+kn}{import} \PY{n+nn}{numpy} \PY{k}{as} \PY{n+nn}{np}
\PY{c+c1}{\PYZsh{}sns.set(style=\PYZsq{}white\PYZsq{}, font\PYZus{}scale=2)}
\PY{o}{\PYZpc{}}\PY{k}{matplotlib} inline
\PY{k+kn}{import} \PY{n+nn}{matplotlib} \PY{k}{as} \PY{n+nn}{mpl}


\PY{n}{mpl}\PY{o}{.}\PY{n}{rcParams}\PY{p}{[}\PY{l+s+s2}{\PYZdq{}}\PY{l+s+s2}{figure.dpi}\PY{l+s+s2}{\PYZdq{}}\PY{p}{]} \PY{o}{=} \PY{l+m+mi}{200}
\PY{k+kn}{from} \PY{n+nn}{matplotlib} \PY{k+kn}{import} \PY{n}{rc}
\PY{n}{rc}\PY{p}{(}\PY{l+s+s1}{\PYZsq{}}\PY{l+s+s1}{font}\PY{l+s+s1}{\PYZsq{}}\PY{p}{,} \PY{n}{family}\PY{o}{=}\PY{l+s+s1}{\PYZsq{}}\PY{l+s+s1}{serif}\PY{l+s+s1}{\PYZsq{}}\PY{p}{)}
\PY{n}{rc}\PY{p}{(}\PY{l+s+s1}{\PYZsq{}}\PY{l+s+s1}{text}\PY{l+s+s1}{\PYZsq{}}\PY{p}{,} \PY{n}{usetex}\PY{o}{=}\PY{k+kc}{True}\PY{p}{)}
\PY{n}{rc}\PY{p}{(}\PY{l+s+s1}{\PYZsq{}}\PY{l+s+s1}{xtick}\PY{l+s+s1}{\PYZsq{}}\PY{p}{,} \PY{n}{labelsize}\PY{o}{=}\PY{l+s+s1}{\PYZsq{}}\PY{l+s+s1}{x\PYZhy{}small}\PY{l+s+s1}{\PYZsq{}}\PY{p}{)}
\PY{n}{rc}\PY{p}{(}\PY{l+s+s1}{\PYZsq{}}\PY{l+s+s1}{ytick}\PY{l+s+s1}{\PYZsq{}}\PY{p}{,} \PY{n}{labelsize}\PY{o}{=}\PY{l+s+s1}{\PYZsq{}}\PY{l+s+s1}{x\PYZhy{}small}\PY{l+s+s1}{\PYZsq{}}\PY{p}{)}

\PY{k}{def} \PY{n+nf}{cm2inch}\PY{p}{(}\PY{n}{value}\PY{p}{)}\PY{p}{:}
    \PY{k}{return} \PY{n}{value}\PY{o}{/}\PY{l+m+mf}{2.54}
\end{Verbatim}
\end{tcolorbox}

    \begin{Verbatim}[commandchars=\\\{\}]
No module named 'pylorenzmie.fitting.cython.cminimizers'
    \end{Verbatim}

    \begin{tcolorbox}[breakable, size=fbox, boxrule=1pt, pad at break*=1mm,colback=cellbackground, colframe=cellborder]
\prompt{In}{incolor}{2}{\boxspacing}
\begin{Verbatim}[commandchars=\\\{\}]
\PY{c+c1}{\PYZsh{}We load the movie}
\PY{n}{vid} \PY{o}{=} \PY{n}{utilities}\PY{o}{.}\PY{n}{video\PYZus{}reader}\PY{p}{(}\PY{l+s+s2}{\PYZdq{}}\PY{l+s+s2}{Basler\PYZus{}acA1920\PYZhy{}155um\PYZus{}\PYZus{}22392621\PYZus{}\PYZus{}20200527\PYZus{}162231224.mp4}\PY{l+s+s2}{\PYZdq{}}\PY{p}{)}
\end{Verbatim}
\end{tcolorbox}

    \begin{tcolorbox}[breakable, size=fbox, boxrule=1pt, pad at break*=1mm,colback=cellbackground, colframe=cellborder]
\prompt{In}{incolor}{3}{\boxspacing}
\begin{Verbatim}[commandchars=\\\{\}]
\PY{c+c1}{\PYZsh{} A function that permits to compute de radial profile of an image this will later be used in order to see if the fits are done correctly}
\PY{k}{def} \PY{n+nf}{radial\PYZus{}profile}\PY{p}{(}\PY{n}{data}\PY{p}{,} \PY{n}{center}\PY{o}{=}\PY{k+kc}{None}\PY{p}{)}\PY{p}{:}
    \PY{k}{if} \PY{n}{center}\PY{o}{==}\PY{k+kc}{None}\PY{p}{:}
        \PY{n}{center} \PY{o}{=} \PY{n}{np}\PY{o}{.}\PY{n}{array}\PY{p}{(}\PY{n}{np}\PY{o}{.}\PY{n}{shape}\PY{p}{(}\PY{n}{data}\PY{p}{)}\PY{p}{)} \PY{o}{/} \PY{l+m+mi}{2}
    
    \PY{n}{y}\PY{p}{,} \PY{n}{x} \PY{o}{=} \PY{n}{np}\PY{o}{.}\PY{n}{indices}\PY{p}{(}\PY{p}{(}\PY{n}{data}\PY{o}{.}\PY{n}{shape}\PY{p}{)}\PY{p}{)}
    \PY{n}{r} \PY{o}{=} \PY{n}{np}\PY{o}{.}\PY{n}{sqrt}\PY{p}{(}\PY{p}{(}\PY{n}{x} \PY{o}{\PYZhy{}} \PY{n}{center}\PY{p}{[}\PY{l+m+mi}{0}\PY{p}{]}\PY{p}{)}\PY{o}{*}\PY{o}{*}\PY{l+m+mi}{2} \PY{o}{+} \PY{p}{(}\PY{n}{y} \PY{o}{\PYZhy{}} \PY{n}{center}\PY{p}{[}\PY{l+m+mi}{1}\PY{p}{]}\PY{p}{)}\PY{o}{*}\PY{o}{*}\PY{l+m+mi}{2}\PY{p}{)}
    \PY{n}{r} \PY{o}{=} \PY{n}{r}\PY{o}{.}\PY{n}{astype}\PY{p}{(}\PY{n+nb}{int}\PY{p}{)}

    \PY{n}{tbin} \PY{o}{=} \PY{n}{np}\PY{o}{.}\PY{n}{bincount}\PY{p}{(}\PY{n}{r}\PY{o}{.}\PY{n}{ravel}\PY{p}{(}\PY{p}{)}\PY{p}{,} \PY{n}{data}\PY{o}{.}\PY{n}{ravel}\PY{p}{(}\PY{p}{)}\PY{p}{)}
    \PY{n}{nr} \PY{o}{=} \PY{n}{np}\PY{o}{.}\PY{n}{bincount}\PY{p}{(}\PY{n}{r}\PY{o}{.}\PY{n}{ravel}\PY{p}{(}\PY{p}{)}\PY{p}{)}
    \PY{n}{radialprofile} \PY{o}{=} \PY{n}{tbin} \PY{o}{/} \PY{n}{nr}
    
    \PY{n}{T} \PY{o}{=} \PY{n}{data}\PY{o}{.}\PY{n}{ravel}\PY{p}{(}\PY{p}{)}
    \PY{n}{V} \PY{o}{=} \PY{n}{r}\PY{o}{.}\PY{n}{ravel}\PY{p}{(}\PY{p}{)}
    
    \PY{n}{err} \PY{o}{=} \PY{p}{[}\PY{n}{np}\PY{o}{.}\PY{n}{std}\PY{p}{(}\PY{n}{T}\PY{p}{[}\PY{n}{V} \PY{o}{==} \PY{n}{u}\PY{p}{]}\PY{p}{)} \PY{k}{for} \PY{n}{u} \PY{o+ow}{in} \PY{n}{np}\PY{o}{.}\PY{n}{unique}\PY{p}{(}\PY{n}{V}\PY{p}{)}\PY{p}{]}
    
    \PY{k}{return} \PY{n}{radialprofile}\PY{p}{,} \PY{n}{err}
\end{Verbatim}
\end{tcolorbox}

    \begin{tcolorbox}[breakable, size=fbox, boxrule=1pt, pad at break*=1mm,colback=cellbackground, colframe=cellborder]
\prompt{In}{incolor}{4}{\boxspacing}
\begin{Verbatim}[commandchars=\\\{\}]
\PY{c+c1}{\PYZsh{} We take a look at the first image of the movie}
\PY{n}{image} \PY{o}{=} \PY{n}{vid}\PY{o}{.}\PY{n}{get\PYZus{}image}\PY{p}{(}\PY{l+m+mi}{1}\PY{p}{)}
\PY{n}{plt}\PY{o}{.}\PY{n}{imshow}\PY{p}{(}\PY{n}{image}\PY{p}{,}\PY{n}{cmap}\PY{o}{=}\PY{l+s+s2}{\PYZdq{}}\PY{l+s+s2}{gray}\PY{l+s+s2}{\PYZdq{}}\PY{p}{)}
\end{Verbatim}
\end{tcolorbox}

            \begin{tcolorbox}[breakable, size=fbox, boxrule=.5pt, pad at break*=1mm, opacityfill=0]
\prompt{Out}{outcolor}{4}{\boxspacing}
\begin{Verbatim}[commandchars=\\\{\}]
<matplotlib.image.AxesImage at 0x1a70b337be0>
\end{Verbatim}
\end{tcolorbox}
        
    \begin{center}
    \adjustimage{max size={0.9\linewidth}{0.9\paperheight}}{example_suivi_particle_files/example_suivi_particle_4_1.png}
    \end{center}
    { \hspace*{\fill} \\}
    
    \begin{tcolorbox}[breakable, size=fbox, boxrule=1pt, pad at break*=1mm,colback=cellbackground, colframe=cellborder]
\prompt{In}{incolor}{5}{\boxspacing}
\begin{Verbatim}[commandchars=\\\{\}]
\PY{c+c1}{\PYZsh{} set the background image (it can also be computed using vid.get\PYZus{}background method)}
\PY{n}{vid}\PY{o}{.}\PY{n}{number} \PY{o}{=} \PY{l+m+mi}{125000}
\PY{n}{vid}\PY{o}{.}\PY{n}{background} \PY{o}{=} \PY{n}{np}\PY{o}{.}\PY{n}{array}\PY{p}{(}\PY{n}{imageio}\PY{o}{.}\PY{n}{imread}\PY{p}{(}\PY{l+s+s2}{\PYZdq{}}\PY{l+s+s2}{background.tiff}\PY{l+s+s2}{\PYZdq{}}\PY{p}{)}\PY{p}{)}
\PY{c+c1}{\PYZsh{}vid.background = vid.get\PYZus{}background(n=50) \PYZsh{} n is the number of image to use to compute the background}
\PY{n}{plt}\PY{o}{.}\PY{n}{imshow}\PY{p}{(}\PY{n}{vid}\PY{o}{.}\PY{n}{background}\PY{p}{,}\PY{n}{cmap}\PY{o}{=}\PY{l+s+s2}{\PYZdq{}}\PY{l+s+s2}{gray}\PY{l+s+s2}{\PYZdq{}}\PY{p}{)}
\end{Verbatim}
\end{tcolorbox}

            \begin{tcolorbox}[breakable, size=fbox, boxrule=.5pt, pad at break*=1mm, opacityfill=0]
\prompt{Out}{outcolor}{5}{\boxspacing}
\begin{Verbatim}[commandchars=\\\{\}]
<matplotlib.image.AxesImage at 0x1a70bc56490>
\end{Verbatim}
\end{tcolorbox}
        
    \begin{center}
    \adjustimage{max size={0.9\linewidth}{0.9\paperheight}}{example_suivi_particle_files/example_suivi_particle_5_1.png}
    \end{center}
    { \hspace*{\fill} \\}
    
    \begin{tcolorbox}[breakable, size=fbox, boxrule=1pt, pad at break*=1mm,colback=cellbackground, colframe=cellborder]
\prompt{In}{incolor}{6}{\boxspacing}
\begin{Verbatim}[commandchars=\\\{\}]
\PY{n}{imageio}\PY{o}{.}\PY{n}{imwrite}\PY{p}{(}\PY{l+s+s2}{\PYZdq{}}\PY{l+s+s2}{background.tiff}\PY{l+s+s2}{\PYZdq{}}\PY{p}{,}\PY{n}{vid}\PY{o}{.}\PY{n}{background}\PY{p}{)}\PY{c+c1}{\PYZsh{} We save the background for possible later use.}
\end{Verbatim}
\end{tcolorbox}

    \begin{tcolorbox}[breakable, size=fbox, boxrule=1pt, pad at break*=1mm,colback=cellbackground, colframe=cellborder]
\prompt{In}{incolor}{7}{\boxspacing}
\begin{Verbatim}[commandchars=\\\{\}]
\PY{c+c1}{\PYZsh{} the normalized image, we can see that their is some movement in the background.}
\PY{c+c1}{\PYZsh{} This could be avoided by computed the background as a function of the time, if the particle diffuses enough.}
\PY{n}{normed\PYZus{}image} \PY{o}{=} \PY{n}{utilities}\PY{o}{.}\PY{n}{normalize}\PY{p}{(}\PY{n}{image}\PY{p}{,}\PY{n}{vid}\PY{o}{.}\PY{n}{background}\PY{p}{)}
\PY{n}{plt}\PY{o}{.}\PY{n}{imshow}\PY{p}{(}\PY{n}{normed\PYZus{}image}\PY{p}{,}\PY{n}{cmap}\PY{o}{=}\PY{l+s+s2}{\PYZdq{}}\PY{l+s+s2}{gray}\PY{l+s+s2}{\PYZdq{}}\PY{p}{)}
\PY{n}{normed\PYZus{}image} \PY{o}{=} \PY{n}{normed\PYZus{}image}
\end{Verbatim}
\end{tcolorbox}

    \begin{center}
    \adjustimage{max size={0.9\linewidth}{0.9\paperheight}}{example_suivi_particle_files/example_suivi_particle_7_0.png}
    \end{center}
    { \hspace*{\fill} \\}
    
    \begin{tcolorbox}[breakable, size=fbox, boxrule=1pt, pad at break*=1mm,colback=cellbackground, colframe=cellborder]
\prompt{In}{incolor}{8}{\boxspacing}
\begin{Verbatim}[commandchars=\\\{\}]
\PY{c+c1}{\PYZsh{} We found the possition of the particle}
\PY{n}{feature} \PY{o}{=} \PY{n}{utilities}\PY{o}{.}\PY{n}{center\PYZus{}find}\PY{p}{(}\PY{n}{image}\PY{p}{)}\PY{p}{[}\PY{l+m+mi}{0}\PY{p}{]}
\PY{n}{utilities}\PY{o}{.}\PY{n}{plot\PYZus{}bounding}\PY{p}{(}\PY{n}{normed\PYZus{}image}\PY{p}{,}\PY{n}{feature}\PY{p}{)}
\end{Verbatim}
\end{tcolorbox}

    \begin{center}
    \adjustimage{max size={0.9\linewidth}{0.9\paperheight}}{example_suivi_particle_files/example_suivi_particle_8_0.png}
    \end{center}
    { \hspace*{\fill} \\}
    
    \hypertarget{fitting-the-first-image}{%
\subsection{Fitting the first image}\label{fitting-the-first-image}}

We fit the first image in order to get the preguess. We first start by
croping the hologram.

    \begin{tcolorbox}[breakable, size=fbox, boxrule=1pt, pad at break*=1mm,colback=cellbackground, colframe=cellborder]
\prompt{In}{incolor}{9}{\boxspacing}
\begin{Verbatim}[commandchars=\\\{\}]
\PY{n}{xc}\PY{p}{,} \PY{n}{yc}\PY{p}{,} \PY{n}{w}\PY{p}{,} \PY{n}{h} \PY{o}{=} \PY{n}{feature}\PY{p}{[}\PY{l+m+mi}{0}\PY{p}{]}
\PY{n}{x\PYZus{}center} \PY{o}{=} \PY{n}{xc}
\PY{n}{y\PYZus{}center} \PY{o}{=} \PY{n}{yc}
\PY{n}{h}\PY{o}{=}\PY{l+m+mi}{200}
\PY{n}{im\PYZus{}c} \PY{o}{=} \PY{n}{fit}\PY{o}{.}\PY{n}{crop}\PY{p}{(}\PY{n}{image}\PY{p}{,} \PY{n+nb}{int}\PY{p}{(}\PY{n}{xc}\PY{p}{)}\PY{p}{,} \PY{n+nb}{int}\PY{p}{(}\PY{n}{yc}\PY{p}{)}\PY{p}{,} \PY{n+nb}{int}\PY{p}{(}\PY{n}{h}\PY{p}{)}\PY{p}{)}
\PY{n}{bk\PYZus{}c} \PY{o}{=} \PY{n}{fit}\PY{o}{.}\PY{n}{crop}\PY{p}{(}\PY{n}{vid}\PY{o}{.}\PY{n}{background}\PY{p}{,} \PY{n+nb}{int}\PY{p}{(}\PY{n}{xc}\PY{p}{)}\PY{p}{,} \PY{n+nb}{int}\PY{p}{(}\PY{n}{yc}\PY{p}{)}\PY{p}{,} \PY{n+nb}{int}\PY{p}{(}\PY{n}{h}\PY{p}{)}\PY{p}{)}
\PY{n}{cropped} \PY{o}{=} \PY{n}{utilities}\PY{o}{.}\PY{n}{normalize}\PY{p}{(}\PY{n}{im\PYZus{}c}\PY{p}{,}\PY{n}{bk\PYZus{}c}\PY{p}{,} \PY{n}{dark\PYZus{}count} \PY{o}{=} \PY{n}{np}\PY{o}{.}\PY{n}{min}\PY{p}{(}\PY{n}{im\PYZus{}c}\PY{p}{)}\PY{p}{)}
\PY{n}{cropped} \PY{o}{=} \PY{n}{cropped} \PY{o}{/} \PY{n}{np}\PY{o}{.}\PY{n}{mean}\PY{p}{(}\PY{n}{cropped}\PY{p}{)}
\PY{n}{plt}\PY{o}{.}\PY{n}{imshow}\PY{p}{(}\PY{n}{cropped}\PY{p}{,}\PY{n}{cmap} \PY{o}{=} \PY{l+s+s2}{\PYZdq{}}\PY{l+s+s2}{gray}\PY{l+s+s2}{\PYZdq{}}\PY{p}{)}
\end{Verbatim}
\end{tcolorbox}

            \begin{tcolorbox}[breakable, size=fbox, boxrule=.5pt, pad at break*=1mm, opacityfill=0]
\prompt{Out}{outcolor}{9}{\boxspacing}
\begin{Verbatim}[commandchars=\\\{\}]
<matplotlib.image.AxesImage at 0x1a71d7e6d00>
\end{Verbatim}
\end{tcolorbox}
        
    \begin{center}
    \adjustimage{max size={0.9\linewidth}{0.9\paperheight}}{example_suivi_particle_files/example_suivi_particle_10_1.png}
    \end{center}
    { \hspace*{\fill} \\}
    
    \begin{tcolorbox}[breakable, size=fbox, boxrule=1pt, pad at break*=1mm,colback=cellbackground, colframe=cellborder]
\prompt{In}{incolor}{10}{\boxspacing}
\begin{Verbatim}[commandchars=\\\{\}]
\PY{c+c1}{\PYZsh{} We setup the fitting method.}
\PY{n}{fitter} \PY{o}{=} \PY{n}{fit}\PY{o}{.}\PY{n}{fitting}\PY{p}{(}\PY{n}{cropped}\PY{p}{,}\PY{l+m+mf}{0.532}\PY{p}{,}\PY{l+m+mf}{0.0513}\PY{p}{)}
\PY{n}{fitter}\PY{o}{.}\PY{n}{make\PYZus{}guess}\PY{p}{(}\PY{l+m+mf}{1.50}\PY{p}{,}\PY{l+m+mf}{1.59}\PY{p}{,}\PY{l+m+mi}{12}\PY{p}{,}\PY{n}{alpha} \PY{o}{=} \PY{l+m+mi}{1}\PY{p}{,}\PY{n}{fit\PYZus{}r}\PY{o}{=}\PY{k+kc}{True}\PY{p}{,} \PY{n}{fit\PYZus{}n}\PY{o}{=}\PY{k+kc}{True}\PY{p}{,}\PY{n}{fit\PYZus{}alpha}\PY{o}{=}\PY{k+kc}{True}\PY{p}{)}
\end{Verbatim}
\end{tcolorbox}

    \begin{tcolorbox}[breakable, size=fbox, boxrule=1pt, pad at break*=1mm,colback=cellbackground, colframe=cellborder]
\prompt{In}{incolor}{11}{\boxspacing}
\begin{Verbatim}[commandchars=\\\{\}]
\PY{c+c1}{\PYZsh{} We do the actual fit.}
\PY{n}{result} \PY{o}{=} \PY{n}{fitter}\PY{o}{.}\PY{n}{fit\PYZus{}single}\PY{p}{(}\PY{n}{cropped}\PY{p}{,} \PY{n}{method} \PY{o}{=} \PY{l+s+s2}{\PYZdq{}}\PY{l+s+s2}{lm}\PY{l+s+s2}{\PYZdq{}}\PY{p}{)}
\end{Verbatim}
\end{tcolorbox}

    \begin{tcolorbox}[breakable, size=fbox, boxrule=1pt, pad at break*=1mm,colback=cellbackground, colframe=cellborder]
\prompt{In}{incolor}{12}{\boxspacing}
\begin{Verbatim}[commandchars=\\\{\}]
\PY{n}{zo} \PY{o}{=} \PY{n}{result}\PY{o}{.}\PY{n}{result}\PY{p}{[}\PY{l+s+s2}{\PYZdq{}}\PY{l+s+s2}{x}\PY{l+s+s2}{\PYZdq{}}\PY{p}{]}\PY{p}{[}\PY{l+m+mi}{2}\PY{p}{]}\PY{o}{*}\PY{l+m+mf}{0.0513}
\PY{n+nb}{print}\PY{p}{(}\PY{n}{result}\PY{o}{.}\PY{n}{result}\PY{p}{[}\PY{l+s+s2}{\PYZdq{}}\PY{l+s+s2}{x}\PY{l+s+s2}{\PYZdq{}}\PY{p}{]}\PY{p}{[}\PY{l+m+mi}{2}\PY{p}{]}\PY{o}{*}\PY{l+m+mf}{0.0513}\PY{p}{)}
\PY{n+nb}{print}\PY{p}{(}\PY{n}{result}\PY{o}{.}\PY{n}{redchi}\PY{p}{)}
\PY{n+nb}{print}\PY{p}{(}\PY{n}{result}\PY{o}{.}\PY{n}{result}\PY{p}{[}\PY{l+s+s2}{\PYZdq{}}\PY{l+s+s2}{x}\PY{l+s+s2}{\PYZdq{}}\PY{p}{]}\PY{p}{)}
\end{Verbatim}
\end{tcolorbox}

    \begin{Verbatim}[commandchars=\\\{\}]
11.427616273713154
7.2459765196825305
[101.23514587 103.00299474 222.76055114   1.5310255    1.58239091
   1.00198476]
    \end{Verbatim}

    We can plot the result to see if the fit worked properly, and, for a
more quantitative comparison we can compute the radial intensity profile
of both hologram and compare them.

    \begin{tcolorbox}[breakable, size=fbox, boxrule=1pt, pad at break*=1mm,colback=cellbackground, colframe=cellborder]
\prompt{In}{incolor}{13}{\boxspacing}
\begin{Verbatim}[commandchars=\\\{\}]
\PY{n}{center} \PY{o}{=} \PY{n}{np}\PY{o}{.}\PY{n}{array}\PY{p}{(}\PY{n}{np}\PY{o}{.}\PY{n}{shape}\PY{p}{(}\PY{n}{fitter}\PY{o}{.}\PY{n}{image}\PY{p}{)}\PY{p}{)}
\end{Verbatim}
\end{tcolorbox}

    \begin{tcolorbox}[breakable, size=fbox, boxrule=1pt, pad at break*=1mm,colback=cellbackground, colframe=cellborder]
\prompt{In}{incolor}{14}{\boxspacing}
\begin{Verbatim}[commandchars=\\\{\}]
\PY{n}{radial\PYZus{}exp}\PY{p}{,} \PY{n}{err} \PY{o}{=} \PY{n}{radial\PYZus{}profile}\PY{p}{(}\PY{n}{fitter}\PY{o}{.}\PY{n}{image}\PY{p}{)}
\PY{n}{theo\PYZus{}exp}\PY{p}{,} \PY{n}{err} \PY{o}{=} \PY{n}{radial\PYZus{}profile}\PY{p}{(}\PY{n}{fitter}\PY{o}{.}\PY{n}{fitter}\PY{o}{.}\PY{n}{model}\PY{o}{.}\PY{n}{hologram}\PY{p}{(}\PY{p}{)}\PY{o}{.}\PY{n}{reshape}\PY{p}{(}\PY{n}{fitter}\PY{o}{.}\PY{n}{shape}\PY{p}{)}\PY{p}{)}
\PY{c+c1}{\PYZsh{} computing first the holgram using the fit resutlt}
\end{Verbatim}
\end{tcolorbox}

    \begin{tcolorbox}[breakable, size=fbox, boxrule=1pt, pad at break*=1mm,colback=cellbackground, colframe=cellborder]
\prompt{In}{incolor}{15}{\boxspacing}
\begin{Verbatim}[commandchars=\\\{\}]
\PY{n}{fit\PYZus{}data} \PY{o}{=} \PY{p}{\PYZob{}}\PY{p}{\PYZcb{}}
\PY{n}{radius\PYZus{}radial} \PY{o}{=} \PY{n}{np}\PY{o}{.}\PY{n}{arange}\PY{p}{(}\PY{n+nb}{len}\PY{p}{(}\PY{n}{radial\PYZus{}exp}\PY{p}{)}\PY{p}{)} \PY{o}{*} \PY{l+m+mf}{0.0513}
\PY{n}{plt}\PY{o}{.}\PY{n}{figure}\PY{p}{(}\PY{n}{figsize} \PY{o}{=} \PY{p}{(}\PY{l+m+mi}{15}\PY{p}{,}\PY{l+m+mi}{15}\PY{p}{)}\PY{p}{)}
\PY{n}{fig} \PY{o}{=} \PY{n}{plt}\PY{o}{.}\PY{n}{figure}\PY{p}{(}\PY{n}{figsize}\PY{o}{=}\PY{p}{(}\PY{n}{cm2inch}\PY{p}{(}\PY{l+m+mf}{8.6}\PY{p}{)}\PY{p}{,}\PY{l+m+mf}{1.65}\PY{o}{*}\PY{n}{cm2inch}\PY{p}{(}\PY{l+m+mf}{8.6}\PY{p}{)}\PY{p}{)}\PY{p}{)}
\PY{n}{fig}\PY{o}{.}\PY{n}{subplots\PYZus{}adjust}\PY{p}{(}\PY{n}{left}\PY{o}{=}\PY{l+m+mf}{0.14}\PY{p}{,} \PY{n}{bottom}\PY{o}{=}\PY{l+m+mf}{.12}\PY{p}{,} \PY{n}{right}\PY{o}{=}\PY{l+m+mf}{.99}\PY{p}{,} \PY{n}{top}\PY{o}{=}\PY{l+m+mf}{.98}\PY{p}{)}

\PY{n}{plt}\PY{o}{.}\PY{n}{subplot}\PY{p}{(}\PY{l+m+mi}{2}\PY{p}{,}\PY{l+m+mi}{2}\PY{p}{,}\PY{l+m+mi}{1}\PY{p}{)}
\PY{n}{plt}\PY{o}{.}\PY{n}{imshow}\PY{p}{(}\PY{n}{fitter}\PY{o}{.}\PY{n}{image}\PY{p}{,} \PY{n}{cmap} \PY{o}{=} \PY{l+s+s2}{\PYZdq{}}\PY{l+s+s2}{gray}\PY{l+s+s2}{\PYZdq{}}\PY{p}{)}
\PY{c+c1}{\PYZsh{}plt.title(\PYZsq{}subplot(2,2,1)\PYZsq{})}

\PY{n}{fit\PYZus{}data}\PY{p}{[}\PY{l+s+s2}{\PYZdq{}}\PY{l+s+s2}{exp\PYZus{}image}\PY{l+s+s2}{\PYZdq{}}\PY{p}{]} \PY{o}{=} \PY{n}{fitter}\PY{o}{.}\PY{n}{image}

\PY{n}{plt}\PY{o}{.}\PY{n}{subplot}\PY{p}{(}\PY{l+m+mi}{2}\PY{p}{,}\PY{l+m+mi}{2}\PY{p}{,}\PY{l+m+mi}{2}\PY{p}{)}
\PY{n}{plt}\PY{o}{.}\PY{n}{imshow}\PY{p}{(}\PY{n}{fitter}\PY{o}{.}\PY{n}{fitter}\PY{o}{.}\PY{n}{model}\PY{o}{.}\PY{n}{hologram}\PY{p}{(}\PY{p}{)}\PY{o}{.}\PY{n}{reshape}\PY{p}{(}\PY{n}{fitter}\PY{o}{.}\PY{n}{shape}\PY{p}{)}\PY{p}{,} \PY{n}{cmap} \PY{o}{=} \PY{l+s+s2}{\PYZdq{}}\PY{l+s+s2}{gray}\PY{l+s+s2}{\PYZdq{}}\PY{p}{)}
\PY{n}{frame1} \PY{o}{=} \PY{n}{plt}\PY{o}{.}\PY{n}{gca}\PY{p}{(}\PY{p}{)}
\PY{n}{frame1}\PY{o}{.}\PY{n}{axes}\PY{o}{.}\PY{n}{yaxis}\PY{o}{.}\PY{n}{set\PYZus{}ticklabels}\PY{p}{(}\PY{p}{[}\PY{p}{]}\PY{p}{)}

\PY{n}{fit\PYZus{}data}\PY{p}{[}\PY{l+s+s2}{\PYZdq{}}\PY{l+s+s2}{th\PYZus{}image}\PY{l+s+s2}{\PYZdq{}}\PY{p}{]} \PY{o}{=} \PY{n}{fitter}\PY{o}{.}\PY{n}{fitter}\PY{o}{.}\PY{n}{model}\PY{o}{.}\PY{n}{hologram}\PY{p}{(}\PY{p}{)}\PY{o}{.}\PY{n}{reshape}\PY{p}{(}\PY{n}{fitter}\PY{o}{.}\PY{n}{shape}\PY{p}{)}


\PY{c+c1}{\PYZsh{}plt.title(\PYZsq{}subplot(2,2,2)\PYZsq{})}

\PY{n}{plt}\PY{o}{.}\PY{n}{subplot}\PY{p}{(}\PY{l+m+mi}{2}\PY{p}{,}\PY{l+m+mi}{2}\PY{p}{,}\PY{p}{(}\PY{l+m+mi}{3}\PY{p}{,}\PY{l+m+mi}{4}\PY{p}{)}\PY{p}{)}
\PY{n}{plt}\PY{o}{.}\PY{n}{plot}\PY{p}{(}\PY{n}{radius\PYZus{}radial}\PY{p}{,} \PY{n}{radial\PYZus{}exp}\PY{p}{,} \PY{n}{label}\PY{o}{=}\PY{l+s+s2}{\PYZdq{}}\PY{l+s+s2}{Experimental}\PY{l+s+s2}{\PYZdq{}}\PY{p}{)}
\PY{n}{plt}\PY{o}{.}\PY{n}{fill\PYZus{}between}\PY{p}{(}\PY{n}{radius\PYZus{}radial}\PY{p}{,} \PY{n}{radial\PYZus{}exp} \PY{o}{\PYZhy{}} \PY{n}{err}\PY{p}{,} \PY{n}{radial\PYZus{}exp} \PY{o}{+} \PY{n}{err}\PY{p}{,} \PY{n}{alpha} \PY{o}{=} \PY{l+m+mf}{0.3}\PY{p}{)}
\PY{n}{plt}\PY{o}{.}\PY{n}{plot}\PY{p}{(}\PY{n}{radius\PYZus{}radial}\PY{p}{,} \PY{n}{theo\PYZus{}exp}\PY{p}{,} \PY{n}{label}\PY{o}{=}\PY{l+s+s2}{\PYZdq{}}\PY{l+s+s2}{Theory}\PY{l+s+s2}{\PYZdq{}}\PY{p}{)}
\PY{n}{plt}\PY{o}{.}\PY{n}{legend}\PY{p}{(}\PY{p}{)}
\PY{n}{plt}\PY{o}{.}\PY{n}{xlabel}\PY{p}{(}\PY{l+s+s2}{\PYZdq{}}\PY{l+s+s2}{radius [pixel]}\PY{l+s+s2}{\PYZdq{}}\PY{p}{)}
\PY{n}{plt}\PY{o}{.}\PY{n}{ylabel}\PY{p}{(}\PY{l+s+s2}{\PYZdq{}}\PY{l+s+s2}{Intensity [a.u.]}\PY{l+s+s2}{\PYZdq{}}\PY{p}{)}

\PY{n}{fit\PYZus{}data}\PY{p}{[}\PY{l+s+s2}{\PYZdq{}}\PY{l+s+s2}{I\PYZus{}r\PYZus{}exp}\PY{l+s+s2}{\PYZdq{}}\PY{p}{]} \PY{o}{=} \PY{n}{radial\PYZus{}exp}
\PY{n}{fit\PYZus{}data}\PY{p}{[}\PY{l+s+s2}{\PYZdq{}}\PY{l+s+s2}{I\PYZus{}errr\PYZus{}exp}\PY{l+s+s2}{\PYZdq{}}\PY{p}{]} \PY{o}{=} \PY{n}{err}

\PY{n}{fit\PYZus{}data}\PY{p}{[}\PY{l+s+s2}{\PYZdq{}}\PY{l+s+s2}{theo\PYZus{}exp}\PY{l+s+s2}{\PYZdq{}}\PY{p}{]} \PY{o}{=} \PY{n}{theo\PYZus{}exp}
\PY{n}{fit\PYZus{}data}\PY{p}{[}\PY{l+s+s2}{\PYZdq{}}\PY{l+s+s2}{I\PYZus{}radius}\PY{l+s+s2}{\PYZdq{}}\PY{p}{]} \PY{o}{=} \PY{n}{radius\PYZus{}radial}

\PY{n}{fig}\PY{o}{.}\PY{n}{set\PYZus{}size\PYZus{}inches}\PY{p}{(}\PY{n}{cm2inch}\PY{p}{(}\PY{l+m+mf}{8.6}\PY{p}{)}\PY{p}{,} \PY{n}{cm2inch}\PY{p}{(}\PY{l+m+mf}{1.6} \PY{o}{*} \PY{l+m+mf}{8.6}\PY{o}{/}\PY{l+m+mf}{1.618}\PY{p}{)}\PY{p}{)}
\PY{n}{plt}\PY{o}{.}\PY{n}{savefig}\PY{p}{(}\PY{l+s+s2}{\PYZdq{}}\PY{l+s+s2}{fit\PYZus{}fig.pdf}\PY{l+s+s2}{\PYZdq{}}\PY{p}{)}
\end{Verbatim}
\end{tcolorbox}

    
    \begin{Verbatim}[commandchars=\\\{\}]
<Figure size 3000x3000 with 0 Axes>
    \end{Verbatim}

    
    \begin{center}
    \adjustimage{max size={0.9\linewidth}{0.9\paperheight}}{example_suivi_particle_files/example_suivi_particle_17_1.png}
    \end{center}
    { \hspace*{\fill} \\}
    
    \begin{tcolorbox}[breakable, size=fbox, boxrule=1pt, pad at break*=1mm,colback=cellbackground, colframe=cellborder]
\prompt{In}{incolor}{16}{\boxspacing}
\begin{Verbatim}[commandchars=\\\{\}]
\PY{n}{fitter}\PY{o}{.}\PY{n}{fit\PYZus{}video}\PY{p}{(}\PY{n}{vid} \PY{o}{=} \PY{n}{vid}\PY{p}{,}\PY{n}{savefile}\PY{o}{=}\PY{l+s+s2}{\PYZdq{}}\PY{l+s+s2}{find\PYZus{}nrfit\PYZus{}result\PYZus{}dur\PYZus{}27052020\PYZus{}n\PYZus{}r\PYZus{}fix\PYZus{}0p0513\PYZus{}wav532.dat}\PY{l+s+s2}{\PYZdq{}}\PY{p}{,}\PY{n}{xc} \PY{o}{=} \PY{n}{x} \PY{p}{,}\PY{n}{yc}\PY{o}{=} \PY{n}{y}\PY{p}{,} \PY{n}{h} \PY{o}{=} \PY{l+m+mi}{200}\PY{p}{,} \PY{n}{n\PYZus{}end}\PY{o}{=}\PY{l+m+mi}{10000}\PY{p}{,}\PY{n}{method} \PY{o}{=} \PY{l+s+s2}{\PYZdq{}}\PY{l+s+s2}{lm}\PY{l+s+s2}{\PYZdq{}}\PY{p}{)}
\end{Verbatim}
\end{tcolorbox}

    \begin{Verbatim}[commandchars=\\\{\}]
100\% 9999/9999 [12:39<00:00, 13.17it/s]
    \end{Verbatim}

    \begin{tcolorbox}[breakable, size=fbox, boxrule=1pt, pad at break*=1mm,colback=cellbackground, colframe=cellborder]
\prompt{In}{incolor}{17}{\boxspacing}
\begin{Verbatim}[commandchars=\\\{\}]
\PY{c+c1}{\PYZsh{} Since the measurement or not saved into the ram we need to load it}
\PY{n}{n\PYZus{}r} \PY{o}{=} \PY{n}{np}\PY{o}{.}\PY{n}{fromfile}\PY{p}{(}\PY{l+s+s1}{\PYZsq{}}\PY{l+s+s1}{find\PYZus{}nr\PYZus{}exame.dat}\PY{l+s+s1}{\PYZsq{}}\PY{p}{,} \PY{n}{dtype}\PY{o}{=}\PY{n}{np}\PY{o}{.}\PY{n}{float64}\PY{p}{)}
\PY{n}{n\PYZus{}r} \PY{o}{=} \PY{n}{n\PYZus{}r}\PY{o}{.}\PY{n}{reshape}\PY{p}{(}\PY{n+nb}{len}\PY{p}{(}\PY{n}{n\PYZus{}r}\PY{p}{)}\PY{o}{/}\PY{o}{/}\PY{l+m+mi}{10}\PY{p}{,}\PY{l+m+mi}{10}\PY{p}{)}
\PY{n}{r} \PY{o}{=} \PY{n}{n\PYZus{}r}\PY{p}{[}\PY{p}{:}\PY{p}{,}\PY{l+m+mi}{3}\PY{p}{]}
\PY{n}{n} \PY{o}{=} \PY{n}{n\PYZus{}r}\PY{p}{[}\PY{p}{:}\PY{p}{,}\PY{l+m+mi}{4}\PY{p}{]}
\end{Verbatim}
\end{tcolorbox}

    \hypertarget{fitting-the-n-r-distributiton-using-a-kde-estimator}{%
\subsection{Fitting the n, r distributiton using a KDE
estimator}\label{fitting-the-n-r-distributiton-using-a-kde-estimator}}

To find the most probable couple of r/n we use a kde estimator using
\texttt{seaborn}

    \begin{tcolorbox}[breakable, size=fbox, boxrule=1pt, pad at break*=1mm,colback=cellbackground, colframe=cellborder]
\prompt{In}{incolor}{18}{\boxspacing}
\begin{Verbatim}[commandchars=\\\{\}]
\PY{k+kn}{import} \PY{n+nn}{numpy} \PY{k}{as} \PY{n+nn}{np}
\PY{k+kn}{import} \PY{n+nn}{scipy}\PY{n+nn}{.}\PY{n+nn}{stats} \PY{k}{as} \PY{n+nn}{st}
\PY{k+kn}{import} \PY{n+nn}{matplotlib}\PY{n+nn}{.}\PY{n+nn}{ticker} \PY{k}{as} \PY{n+nn}{ticker}

\PY{n}{data} \PY{o}{=} \PY{n}{np}\PY{o}{.}\PY{n}{random}\PY{o}{.}\PY{n}{multivariate\PYZus{}normal}\PY{p}{(}\PY{p}{(}\PY{l+m+mi}{0}\PY{p}{,} \PY{l+m+mi}{0}\PY{p}{)}\PY{p}{,} \PY{p}{[}\PY{p}{[}\PY{l+m+mf}{0.8}\PY{p}{,} \PY{l+m+mf}{0.05}\PY{p}{]}\PY{p}{,} \PY{p}{[}\PY{l+m+mf}{0.05}\PY{p}{,} \PY{l+m+mf}{0.7}\PY{p}{]}\PY{p}{]}\PY{p}{,} \PY{l+m+mi}{100}\PY{p}{)}
\PY{n}{x} \PY{o}{=} \PY{n}{r}\PY{p}{[}\PY{p}{(}\PY{n}{r}\PY{o}{\PYZgt{}}\PY{l+m+mf}{1.5}\PY{p}{)} \PY{o}{\PYZam{}} \PY{p}{(}\PY{n}{r}\PY{o}{\PYZlt{}}\PY{l+m+mf}{1.555}\PY{p}{)}\PY{p}{]}
\PY{n}{y} \PY{o}{=} \PY{n}{n}\PY{p}{[}\PY{p}{(}\PY{n}{r}\PY{o}{\PYZgt{}}\PY{l+m+mf}{1.5}\PY{p}{)} \PY{o}{\PYZam{}} \PY{p}{(}\PY{n}{r}\PY{o}{\PYZlt{}}\PY{l+m+mf}{1.555}\PY{p}{)}\PY{p}{]}
\PY{n}{xmin}\PY{p}{,} \PY{n}{xmax} \PY{o}{=} \PY{n}{np}\PY{o}{.}\PY{n}{min}\PY{p}{(}\PY{n}{x}\PY{p}{)}\PY{p}{,} \PY{n}{np}\PY{o}{.}\PY{n}{max}\PY{p}{(}\PY{n}{x}\PY{p}{)}
\PY{n}{ymin}\PY{p}{,} \PY{n}{ymax} \PY{o}{=} \PY{n}{np}\PY{o}{.}\PY{n}{min}\PY{p}{(}\PY{n}{y}\PY{p}{)}\PY{p}{,} \PY{n}{np}\PY{o}{.}\PY{n}{max}\PY{p}{(}\PY{n}{y}\PY{p}{)}

\PY{c+c1}{\PYZsh{} Peform the kernel density estimate}
\PY{n}{xx}\PY{p}{,} \PY{n}{yy} \PY{o}{=} \PY{n}{np}\PY{o}{.}\PY{n}{mgrid}\PY{p}{[}\PY{n}{xmin}\PY{p}{:}\PY{n}{xmax}\PY{p}{:}\PY{l+m+mi}{100}\PY{n}{j}\PY{p}{,} \PY{n}{ymin}\PY{p}{:}\PY{n}{ymax}\PY{p}{:}\PY{l+m+mi}{100}\PY{n}{j}\PY{p}{]}
\PY{n}{positions} \PY{o}{=} \PY{n}{np}\PY{o}{.}\PY{n}{vstack}\PY{p}{(}\PY{p}{[}\PY{n}{xx}\PY{o}{.}\PY{n}{ravel}\PY{p}{(}\PY{p}{)}\PY{p}{,} \PY{n}{yy}\PY{o}{.}\PY{n}{ravel}\PY{p}{(}\PY{p}{)}\PY{p}{]}\PY{p}{)}
\PY{n}{values} \PY{o}{=} \PY{n}{np}\PY{o}{.}\PY{n}{vstack}\PY{p}{(}\PY{p}{[}\PY{n}{x}\PY{p}{,} \PY{n}{y}\PY{p}{]}\PY{p}{)}
\PY{n}{kernel} \PY{o}{=} \PY{n}{st}\PY{o}{.}\PY{n}{gaussian\PYZus{}kde}\PY{p}{(}\PY{n}{values}\PY{p}{)}
\PY{n}{f} \PY{o}{=} \PY{n}{np}\PY{o}{.}\PY{n}{reshape}\PY{p}{(}\PY{n}{kernel}\PY{p}{(}\PY{n}{positions}\PY{p}{)}\PY{o}{.}\PY{n}{T}\PY{p}{,} \PY{n}{xx}\PY{o}{.}\PY{n}{shape}\PY{p}{)}
\PY{n}{f} \PY{o}{=} \PY{n}{f}\PY{o}{/}\PY{n}{np}\PY{o}{.}\PY{n}{max}\PY{p}{(}\PY{n}{f}\PY{p}{)}
\end{Verbatim}
\end{tcolorbox}

    \begin{tcolorbox}[breakable, size=fbox, boxrule=1pt, pad at break*=1mm,colback=cellbackground, colframe=cellborder]
\prompt{In}{incolor}{19}{\boxspacing}
\begin{Verbatim}[commandchars=\\\{\}]
\PY{n}{np}\PY{o}{.}\PY{n}{round}\PY{p}{(}\PY{n}{np}\PY{o}{.}\PY{n}{max}\PY{p}{(}\PY{n}{f}\PY{p}{)}\PY{p}{)}
\end{Verbatim}
\end{tcolorbox}

            \begin{tcolorbox}[breakable, size=fbox, boxrule=.5pt, pad at break*=1mm, opacityfill=0]
\prompt{Out}{outcolor}{19}{\boxspacing}
\begin{Verbatim}[commandchars=\\\{\}]
1.0
\end{Verbatim}
\end{tcolorbox}
        
    \begin{tcolorbox}[breakable, size=fbox, boxrule=1pt, pad at break*=1mm,colback=cellbackground, colframe=cellborder]
\prompt{In}{incolor}{20}{\boxspacing}
\begin{Verbatim}[commandchars=\\\{\}]
\PY{n}{fig} \PY{o}{=} \PY{n}{plt}\PY{o}{.}\PY{n}{figure}\PY{p}{(}\PY{p}{)}
\PY{n}{fig}\PY{o}{.}\PY{n}{subplots\PYZus{}adjust}\PY{p}{(}\PY{n}{left}\PY{o}{=}\PY{l+m+mf}{0.16}\PY{p}{,} \PY{n}{bottom}\PY{o}{=}\PY{l+m+mf}{.20}\PY{p}{,} \PY{n}{right}\PY{o}{=}\PY{l+m+mf}{.99}\PY{p}{,} \PY{n}{top}\PY{o}{=}\PY{l+m+mf}{.99}\PY{p}{)}
\PY{n}{ax} \PY{o}{=} \PY{n}{fig}\PY{o}{.}\PY{n}{gca}\PY{p}{(}\PY{p}{)}
\PY{c+c1}{\PYZsh{}ax.set\PYZus{}xlim(1.505, 1.53)}
\PY{c+c1}{\PYZsh{}ax.set\PYZus{}ylim(1.575, 1.6)}
\PY{c+c1}{\PYZsh{} Contourf plot}
\PY{n}{cfset} \PY{o}{=} \PY{n}{ax}\PY{o}{.}\PY{n}{contourf}\PY{p}{(}\PY{n}{xx}\PY{p}{,} \PY{n}{yy}\PY{p}{,} \PY{n}{f}\PY{p}{,} \PY{n}{cmap}\PY{o}{=}\PY{l+s+s1}{\PYZsq{}}\PY{l+s+s1}{Blues}\PY{l+s+s1}{\PYZsq{}}\PY{p}{)}
\PY{c+c1}{\PYZsh{}\PYZsh{} Or kernel density estimate plot instead of the contourf plot}
\PY{c+c1}{\PYZsh{}ax.imshow(np.rot90(f), cmap=\PYZsq{}Blues\PYZsq{}, extent=[xmin, xmax, ymin, ymax])}
\PY{c+c1}{\PYZsh{} Contour plot}
\PY{n}{cset} \PY{o}{=} \PY{n}{ax}\PY{o}{.}\PY{n}{contour}\PY{p}{(}\PY{n}{xx}\PY{p}{,} \PY{n}{yy}\PY{p}{,} \PY{n}{f}\PY{p}{,} \PY{n}{colors}\PY{o}{=}\PY{l+s+s1}{\PYZsq{}}\PY{l+s+s1}{k}\PY{l+s+s1}{\PYZsq{}}\PY{p}{,} \PY{n}{levels}\PY{o}{=}\PY{l+m+mi}{6}\PY{p}{)}
\PY{c+c1}{\PYZsh{} Label plot}
\PY{n}{ax}\PY{o}{.}\PY{n}{xaxis}\PY{o}{.}\PY{n}{set\PYZus{}major\PYZus{}formatter}\PY{p}{(}\PY{n}{ticker}\PY{o}{.}\PY{n}{FormatStrFormatter}\PY{p}{(}\PY{l+s+s1}{\PYZsq{}}\PY{l+s+si}{\PYZpc{}1.3f}\PY{l+s+s1}{\PYZsq{}}\PY{p}{)}\PY{p}{)}
\PY{n}{ax}\PY{o}{.}\PY{n}{yaxis}\PY{o}{.}\PY{n}{set\PYZus{}major\PYZus{}formatter}\PY{p}{(}\PY{n}{ticker}\PY{o}{.}\PY{n}{FormatStrFormatter}\PY{p}{(}\PY{l+s+s1}{\PYZsq{}}\PY{l+s+si}{\PYZpc{}1.3f}\PY{l+s+s1}{\PYZsq{}}\PY{p}{)}\PY{p}{)}
\PY{n}{ax}\PY{o}{.}\PY{n}{clabel}\PY{p}{(}\PY{n}{cset}\PY{p}{,} \PY{n}{inline}\PY{o}{=}\PY{l+m+mi}{1}\PY{p}{,} \PY{n}{fontsize}\PY{o}{=}\PY{l+m+mi}{10}\PY{p}{,} \PY{n}{fmt}\PY{o}{=}\PY{l+s+s2}{\PYZdq{}}\PY{l+s+si}{\PYZpc{}1.1f}\PY{l+s+s2}{\PYZdq{}}\PY{p}{)}
\PY{n}{plt}\PY{o}{.}\PY{n}{scatter}\PY{p}{(}\PY{n}{xx}\PY{p}{[}\PY{n}{np}\PY{o}{.}\PY{n}{where}\PY{p}{(}\PY{n}{f} \PY{o}{==} \PY{l+m+mi}{1}\PY{p}{)}\PY{p}{]}\PY{p}{,}\PY{n}{yy}\PY{p}{[}\PY{n}{np}\PY{o}{.}\PY{n}{where}\PY{p}{(}\PY{n}{f} \PY{o}{==} \PY{l+m+mi}{1}\PY{p}{)}\PY{p}{]}\PY{p}{,} \PY{n}{color} \PY{o}{=} \PY{l+s+s2}{\PYZdq{}}\PY{l+s+s2}{red}\PY{l+s+s2}{\PYZdq{}}\PY{p}{,} \PY{n}{marker}\PY{o}{=}\PY{l+s+s2}{\PYZdq{}}\PY{l+s+s2}{+}\PY{l+s+s2}{\PYZdq{}}\PY{p}{)}
\PY{n}{ax}\PY{o}{.}\PY{n}{set\PYZus{}xlabel}\PY{p}{(}\PY{l+s+s2}{\PYZdq{}}\PY{l+s+s2}{radius (\PYZdl{}}\PY{l+s+s2}{\PYZbs{}}\PY{l+s+s2}{mathrm}\PY{l+s+s2}{\PYZob{}}\PY{l+s+s2}{\PYZbs{}}\PY{l+s+s2}{mu m\PYZcb{}\PYZdl{})}\PY{l+s+s2}{\PYZdq{}}\PY{p}{)}
\PY{n}{ax}\PY{o}{.}\PY{n}{set\PYZus{}ylabel}\PY{p}{(}\PY{l+s+s2}{\PYZdq{}}\PY{l+s+s2}{\PYZdl{}n\PYZus{}p\PYZdl{}}\PY{l+s+s2}{\PYZdq{}}\PY{p}{)}
\PY{c+c1}{\PYZsh{}plt.title(\PYZdq{}KDE r n\PYZdq{})}
\PY{n}{fig}\PY{o}{.}\PY{n}{set\PYZus{}size\PYZus{}inches}\PY{p}{(}\PY{n}{cm2inch}\PY{p}{(}\PY{l+m+mi}{16}\PY{p}{)}\PY{p}{,} \PY{n}{cm2inch}\PY{p}{(}\PY{l+m+mf}{9.9}\PY{p}{)}\PY{p}{)}

\PY{n}{plt}\PY{o}{.}\PY{n}{tight\PYZus{}layout}\PY{p}{(}\PY{p}{)}
\PY{n}{fig}\PY{o}{.}\PY{n}{savefig}\PY{p}{(}\PY{l+s+s1}{\PYZsq{}}\PY{l+s+s1}{KDErn.pdf}\PY{l+s+s1}{\PYZsq{}}\PY{p}{)}
\PY{c+c1}{\PYZsh{}pl.show()}
\end{Verbatim}
\end{tcolorbox}

    \begin{center}
    \adjustimage{max size={0.9\linewidth}{0.9\paperheight}}{example_suivi_particle_files/example_suivi_particle_23_0.png}
    \end{center}
    { \hspace*{\fill} \\}
    
    \begin{tcolorbox}[breakable, size=fbox, boxrule=1pt, pad at break*=1mm,colback=cellbackground, colframe=cellborder]
\prompt{In}{incolor}{21}{\boxspacing}
\begin{Verbatim}[commandchars=\\\{\}]
\PY{n+nb}{print}\PY{p}{(}\PY{l+s+s2}{\PYZdq{}}\PY{l+s+s2}{ n determined with : mu=}\PY{l+s+si}{\PYZob{}0\PYZcb{}}\PY{l+s+s2}{, sigma=}\PY{l+s+si}{\PYZob{}1\PYZcb{}}\PY{l+s+s2}{\PYZdq{}}\PY{o}{.}\PY{n}{format}\PY{p}{(}\PY{n}{np}\PY{o}{.}\PY{n}{mean}\PY{p}{(}\PY{n}{yy}\PY{p}{[}\PY{n}{np}\PY{o}{.}\PY{n}{where}\PY{p}{(}\PY{n}{f} \PY{o}{\PYZgt{}} \PY{l+m+mf}{0.1}\PY{p}{)}\PY{p}{]}\PY{p}{)}\PY{p}{,} \PY{n}{np}\PY{o}{.}\PY{n}{std}\PY{p}{(}\PY{n}{yy}\PY{p}{[}\PY{n}{np}\PY{o}{.}\PY{n}{where}\PY{p}{(}\PY{n}{f} \PY{o}{\PYZgt{}} \PY{l+m+mf}{0.1}\PY{p}{)}\PY{p}{]}\PY{p}{)}\PY{p}{)}\PY{p}{)}
\PY{n+nb}{print}\PY{p}{(}\PY{l+s+s2}{\PYZdq{}}\PY{l+s+s2}{ r determined with : mu=}\PY{l+s+si}{\PYZob{}0\PYZcb{}}\PY{l+s+s2}{, sigma=}\PY{l+s+si}{\PYZob{}1\PYZcb{}}\PY{l+s+s2}{\PYZdq{}}\PY{o}{.}\PY{n}{format}\PY{p}{(}\PY{n}{np}\PY{o}{.}\PY{n}{mean}\PY{p}{(}\PY{n}{xx}\PY{p}{[}\PY{n}{np}\PY{o}{.}\PY{n}{where}\PY{p}{(}\PY{n}{f} \PY{o}{\PYZgt{}} \PY{l+m+mf}{0.1}\PY{p}{)}\PY{p}{]}\PY{p}{)}\PY{p}{,} \PY{n}{np}\PY{o}{.}\PY{n}{std}\PY{p}{(}\PY{n}{xx}\PY{p}{[}\PY{n}{np}\PY{o}{.}\PY{n}{where}\PY{p}{(}\PY{n}{f} \PY{o}{\PYZgt{}} \PY{l+m+mf}{0.1}\PY{p}{)}\PY{p}{]}\PY{p}{)}\PY{p}{)}\PY{p}{)}
\end{Verbatim}
\end{tcolorbox}

    \begin{Verbatim}[commandchars=\\\{\}]
 n determined with : mu=1.5851200393768743, sigma=0.003267685282504072
 r determined with : mu=1.5181266656310368, sigma=0.00682411690457934
    \end{Verbatim}

    \begin{tcolorbox}[breakable, size=fbox, boxrule=1pt, pad at break*=1mm,colback=cellbackground, colframe=cellborder]
\prompt{In}{incolor}{22}{\boxspacing}
\begin{Verbatim}[commandchars=\\\{\}]
\PY{p}{(}\PY{n}{mu\PYZus{}n}\PY{p}{,} \PY{n}{mu\PYZus{}r}\PY{p}{)} \PY{o}{=} \PY{n}{np}\PY{o}{.}\PY{n}{mean}\PY{p}{(}\PY{n}{yy}\PY{p}{[}\PY{n}{np}\PY{o}{.}\PY{n}{where}\PY{p}{(}\PY{n}{f} \PY{o}{\PYZgt{}} \PY{l+m+mf}{0.1}\PY{p}{)}\PY{p}{]}\PY{p}{)} \PY{p}{,}  \PY{n}{np}\PY{o}{.}\PY{n}{mean}\PY{p}{(}\PY{n}{xx}\PY{p}{[}\PY{n}{np}\PY{o}{.}\PY{n}{where}\PY{p}{(}\PY{n}{f} \PY{o}{\PYZgt{}} \PY{l+m+mf}{0.1}\PY{p}{)}\PY{p}{]}\PY{p}{)}
\end{Verbatim}
\end{tcolorbox}

    \hypertarget{fitting-the-whole-movie}{%
\subsection{Fitting the whole movie}\label{fitting-the-whole-movie}}

Now that the measurement of n and r is one we can move on the
measurement of the whole trajectory by simply using
\texttt{fitter.fit\_video}. For demonstration purposes, I only fit here
at \(\simeq 22\) image per seconds, if can goes up to at least \(60\)
with recent GPU.

    \begin{tcolorbox}[breakable, size=fbox, boxrule=1pt, pad at break*=1mm,colback=cellbackground, colframe=cellborder]
\prompt{In}{incolor}{23}{\boxspacing}
\begin{Verbatim}[commandchars=\\\{\}]
\PY{k}{del} \PY{n}{fitter}
\PY{n}{fitter} \PY{o}{=} \PY{n}{fit}\PY{o}{.}\PY{n}{fitting}\PY{p}{(}\PY{n}{cropped}\PY{p}{,}\PY{l+m+mf}{0.532}\PY{p}{,}\PY{l+m+mf}{0.0513}\PY{p}{)}
\PY{n}{fitter}\PY{o}{.}\PY{n}{make\PYZus{}guess}\PY{p}{(}\PY{n}{mu\PYZus{}r}\PY{p}{,}\PY{n}{mu\PYZus{}n}\PY{p}{,}\PY{n}{zo}\PY{p}{,}\PY{n}{alpha} \PY{o}{=} \PY{l+m+mi}{1}\PY{p}{,}\PY{n}{fit\PYZus{}r}\PY{o}{=}\PY{k+kc}{False}\PY{p}{,} \PY{n}{fit\PYZus{}n}\PY{o}{=}\PY{k+kc}{False}\PY{p}{,}\PY{n}{fit\PYZus{}alpha}\PY{o}{=}\PY{k+kc}{False}\PY{p}{)}
\PY{c+c1}{\PYZsh{}result = fitter.fit\PYZus{}single(cropped, method = \PYZdq{}lm\PYZdq{})}
\PY{n}{fitter}\PY{o}{.}\PY{n}{fit\PYZus{}video}\PY{p}{(}\PY{n}{vid} \PY{o}{=} \PY{n}{vid}\PY{p}{,}\PY{n}{savefile}\PY{o}{=}\PY{l+s+s2}{\PYZdq{}}\PY{l+s+s2}{fit\PYZus{}result\PYZus{}1p5kPa\PYZus{}18122019\PYZus{}n\PYZus{}r\PYZus{}fix\PYZus{}0p0883\PYZus{}wav\PYZus{}532\PYZus{}ex1.dat}\PY{l+s+s2}{\PYZdq{}}\PY{p}{,}\PY{n}{xc} \PY{o}{=} \PY{n}{xc} \PY{p}{,}\PY{n}{yc}\PY{o}{=} \PY{n}{yc}\PY{p}{,} \PY{n}{h} \PY{o}{=} \PY{l+m+mi}{200}\PY{p}{,} \PY{n}{n\PYZus{}end}\PY{o}{=}\PY{l+m+mi}{10000}\PY{p}{,}\PY{n}{method} \PY{o}{=} \PY{l+s+s2}{\PYZdq{}}\PY{l+s+s2}{lm}\PY{l+s+s2}{\PYZdq{}}\PY{p}{)}
\end{Verbatim}
\end{tcolorbox}

    \begin{Verbatim}[commandchars=\\\{\}]
100\% 9999/9999 [07:24<00:00, 22.47it/s]
    \end{Verbatim}

    \begin{tcolorbox}[breakable, size=fbox, boxrule=1pt, pad at break*=1mm,colback=cellbackground, colframe=cellborder]
\prompt{In}{incolor}{24}{\boxspacing}
\begin{Verbatim}[commandchars=\\\{\}]
\PY{k+kn}{import} \PY{n+nn}{numpy} \PY{k}{as} \PY{n+nn}{np}
\PY{n}{data} \PY{o}{=} \PY{n}{np}\PY{o}{.}\PY{n}{fromfile}\PY{p}{(}\PY{l+s+s1}{\PYZsq{}}\PY{l+s+s1}{fit\PYZus{}result\PYZus{}1p5kPa\PYZus{}18122019\PYZus{}n\PYZus{}r\PYZus{}fix\PYZus{}0p0883\PYZus{}wav\PYZus{}532\PYZus{}ex1.dat}\PY{l+s+s1}{\PYZsq{}}\PY{p}{,} \PY{n}{dtype}\PY{o}{=}\PY{n}{np}\PY{o}{.}\PY{n}{float64}\PY{p}{)}
\PY{n}{data} \PY{o}{=} \PY{n}{data}\PY{o}{.}\PY{n}{reshape}\PY{p}{(}\PY{n+nb}{len}\PY{p}{(}\PY{n}{data}\PY{p}{)}\PY{o}{/}\PY{o}{/}\PY{l+m+mi}{10}\PY{p}{,}\PY{l+m+mi}{10}\PY{p}{)}
\PY{n}{x} \PY{o}{=} \PY{n}{data}\PY{p}{[}\PY{p}{:}\PY{p}{,}\PY{l+m+mi}{0}\PY{p}{]}\PY{o}{*}\PY{l+m+mf}{0.0513}
\PY{n}{y} \PY{o}{=} \PY{n}{data}\PY{p}{[}\PY{p}{:}\PY{p}{,}\PY{l+m+mi}{1}\PY{p}{]}\PY{o}{*}\PY{l+m+mf}{0.0513}
\PY{n}{z} \PY{o}{=} \PY{n}{data}\PY{p}{[}\PY{p}{:}\PY{p}{,}\PY{l+m+mi}{2}\PY{p}{]}\PY{o}{*}\PY{l+m+mf}{0.0513}
\end{Verbatim}
\end{tcolorbox}
\newpage

    \hypertarget{plot-the-trajectory}{%
\subsection{Plot the trajectory}\label{plot-the-trajectory}}

    \begin{tcolorbox}[breakable, size=fbox, boxrule=1pt, pad at break*=1mm,colback=cellbackground, colframe=cellborder]
\prompt{In}{incolor}{25}{\boxspacing}
\begin{Verbatim}[commandchars=\\\{\}]
\PY{n}{plt}\PY{o}{.}\PY{n}{plot}\PY{p}{(}\PY{n}{np}\PY{o}{.}\PY{n}{arange}\PY{p}{(}\PY{n+nb}{len}\PY{p}{(}\PY{n}{z}\PY{p}{)}\PY{p}{)}\PY{o}{/}\PY{l+m+mi}{60}\PY{p}{,} \PY{n}{x}\PY{p}{)}
\PY{n}{plt}\PY{o}{.}\PY{n}{plot}\PY{p}{(}\PY{n}{np}\PY{o}{.}\PY{n}{arange}\PY{p}{(}\PY{n+nb}{len}\PY{p}{(}\PY{n}{z}\PY{p}{)}\PY{p}{)}\PY{o}{/}\PY{l+m+mi}{60}\PY{p}{,} \PY{n}{y}\PY{p}{)}
\PY{n}{plt}\PY{o}{.}\PY{n}{ylabel}\PY{p}{(}\PY{l+s+s2}{\PYZdq{}}\PY{l+s+s2}{position \PYZdl{}}\PY{l+s+s2}{\PYZbs{}}\PY{l+s+s2}{mathrm}\PY{l+s+s2}{\PYZob{}}\PY{l+s+s2}{\PYZbs{}}\PY{l+s+s2}{mu m\PYZcb{}\PYZdl{}}\PY{l+s+s2}{\PYZdq{}}\PY{p}{)}
\PY{n}{plt}\PY{o}{.}\PY{n}{xlabel}\PY{p}{(}\PY{l+s+s2}{\PYZdq{}}\PY{l+s+s2}{time (s)}\PY{l+s+s2}{\PYZdq{}}\PY{p}{)}
\end{Verbatim}
\end{tcolorbox}

            \begin{tcolorbox}[breakable, size=fbox, boxrule=.5pt, pad at break*=1mm, opacityfill=0]
\prompt{Out}{outcolor}{25}{\boxspacing}
\begin{Verbatim}[commandchars=\\\{\}]
Text(0.5, 0, 'time (s)')
\end{Verbatim}
\end{tcolorbox}
        
    \begin{center}
    \adjustimage{max size={0.9\linewidth}{0.9\paperheight}}{example_suivi_particle_files/example_suivi_particle_30_1.png}
    \end{center}
    { \hspace*{\fill} \\}
    
    \begin{tcolorbox}[breakable, size=fbox, boxrule=1pt, pad at break*=1mm,colback=cellbackground, colframe=cellborder]
\prompt{In}{incolor}{26}{\boxspacing}
\begin{Verbatim}[commandchars=\\\{\}]
\PY{n}{plt}\PY{o}{.}\PY{n}{plot}\PY{p}{(}\PY{n}{np}\PY{o}{.}\PY{n}{arange}\PY{p}{(}\PY{n+nb}{len}\PY{p}{(}\PY{n}{z}\PY{p}{)}\PY{p}{)}\PY{o}{/}\PY{l+m+mi}{60}\PY{p}{,} \PY{n}{z}\PY{p}{)}
\PY{n}{plt}\PY{o}{.}\PY{n}{ylabel}\PY{p}{(}\PY{l+s+s2}{\PYZdq{}}\PY{l+s+s2}{height \PYZdl{}}\PY{l+s+s2}{\PYZbs{}}\PY{l+s+s2}{mathrm}\PY{l+s+s2}{\PYZob{}}\PY{l+s+s2}{\PYZbs{}}\PY{l+s+s2}{mu m\PYZcb{}\PYZdl{}}\PY{l+s+s2}{\PYZdq{}}\PY{p}{)}
\PY{n}{plt}\PY{o}{.}\PY{n}{xlabel}\PY{p}{(}\PY{l+s+s2}{\PYZdq{}}\PY{l+s+s2}{time (s)}\PY{l+s+s2}{\PYZdq{}}\PY{p}{)}
\end{Verbatim}
\end{tcolorbox}

\newpage

            \begin{tcolorbox}[breakable, size=fbox, boxrule=.5pt, pad at break*=1mm, opacityfill=0]
\prompt{Out}{outcolor}{26}{\boxspacing}
\begin{Verbatim}[commandchars=\\\{\}]
Text(0.5, 0, 'time (s)')
\end{Verbatim}
\end{tcolorbox}
        
    \begin{center}
    \adjustimage{max size={0.9\linewidth}{0.9\paperheight}}{example_suivi_particle_files/example_suivi_particle_31_1.png}
    \end{center}
    { \hspace*{\fill} \\}
    

    % Add a bibliography block to the postdoc
    
    
    
\end{document}
